\chapter{Of Kveldulf and his sons}
There was a man named Ulf, son of Bjalf, and Hallbera, daughter of Ulf the fearless; she was sister of Hallbjorn Half-giant in Hrafnista, and he the father of Kettle H\ae ing. Ulf was a man so tall and strong that none could match him, and in his youth he roved the seas as a freebooter. In fellowship with him was one Kari of Berdla, a man of renown for strength and daring; he was a Berserk. Ulf and he had one common purse, and were the dearest friends.

But when they gave up freebooting, Kari went to his estate at Berdla, being a man of great wealth. Three children had Kari, one son named Eyvind Lambi, another Aulvir Hnuf, and a daughter Salbjorg, who was a most beautiful woman of a noble spirit. Her did Ulf take to wife, and then he too went to his estates. Wealthy he was both in lands and chattels; he took baron's rank as his forefathers had done, and became a great man. It was told of Ulf that he was a great householder; it was his wont to rise up early, and then go round among his labourers or where his smiths were, and to overlook his stalk and fields, and at times he would talk with such as needed his counsel, and good counsel he could give in all things, for he was very wise. But everyday as evening drew on he became sullen, so that few could come to speak with him. He was an evening sleeper, and it was commonly said that he was very shape strong. He was called Kveldulf.

Kveldulf and his wife had two sons, the elder was named Thorolf, the younger Grim; these, when they grew up, were both tall men and strong, as was their father. But Thorolf was most comely as well as doughty, favoring his mother's kin; very cheery was he, liberal, impetuous in everything, a good trader, winning the hearts of all men. Grim was swarthy, ill-favoured, like his father both in face and mind; he became a good man of business; skilful was he in wood and iron, an excellent smith. In the winter he often went to the herring fishing, and with him many house-carles.

But when Thorolf was twenty years old, then he made him ready to go a harrying. Kveldulf gave him a long-ship, and Kari of Berdla's sons, Eyvind and Aulvir, resolved to go on that voyage, taking a large force and another long-ship; and they roved the seas in the summer, and got them wealth, and had a large booty to divide. For several summers they were out roving, but stayed at home in winter with their fathers. Thorolf brought home many costly things, and took them to his father and mother; thus they were well-to-do both for possessions and honour. Kveldulf was now well stricken in years, and his sons were grown men.

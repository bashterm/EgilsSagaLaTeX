\chapter{Thorolf in Finmark}
In the winter Thorolf took his way up to the fells with a large force of not less than ninety men, whereas before it had been the wont of the king's stewards to have thirty men, and sometimes fewer. He took with him plenty of wares for trading. At once he appointed a meeting with the Finns, took of them the tribute, and held a fair with them. All was managed with goodwill and friendship, though not without fear on the Finns' side. Far and wide about Finmark did he travel; but when he reached the fells eastward, he heard that the Kylfings were come from the east, and were there for trading with the Finns, but in some places for plunder also. Thorolf set Finns to spy out the movements of the Kylfings, and he followed after to search for them, and came upon thirty men in one den, all of whom he slew, letting none escape. Afterwards he found together fifteen or twenty. In all they slew near upon a hundred, and took immense booty, and returned in the spring after doing this.

Thorolf then went to his estates at Sandness, and remained there through the spring. He had a long-ship built, large, and with a dragon's head, fitted out in the best style; this he took with him from the north. Thorolf gathered great stores of what there was in Halogaland, employing his men after the herrings and in other fishing; seal-hunting there was too in abundance, and egg-gathering, and all such provision he had brought to him. Never had he fewer freedmen about his home than a hundred; he was open-handed and liberal, and readily made friends with the great, and with all that were near him. A mighty man he became, and he bestowed much care on his ships, equipment, and weapons.

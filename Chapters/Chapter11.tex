\chapter{The king feasts with Thorolf}
King Harold went that summer to Halogaland, and banquets were made ready against his coming, both where his estates were, and also by barons and powerful landowners. Thorolf prepared a banquet for the king at great cost; it was fixed for when the king should come there. To this he bade a numerous company, the best men that could be found. The king had about three hundred men with him when he came to the banquet, but Thorolf had five hundred present. Thorolf had caused a large granary to be fitted up where the drinking should be, for there was no hall large enough to contain all that multitude. And all around the building shields were hung.

The king sate in the high seat; but when the foremost bench was filled, then the king looked round, and he turned red, but spoke not, and men thought they could see he was angry. The banquet was magnificent, and all the viands of the best. The king, however, was gloomy; he remained there three nights, as had been intended. On the day when the king was to leave Thorolf went to him, and offered that they should go together down to the strand. The king did so, and there, moored off the land, floated that dragon-ship which Thorolf had had built, with tent and tackling complete. Thorolf gave the ship to the king, and prayed the king to believe that he had gathered such numbers for this end, to show the king honour, and not to enter into rivalry with him. The king took Thorolf's words well, and then became merry and cheerful. Many added their good word, saying (as was true) that the banquet was most splendid, and the farewell escort magnificent, and that the king gained much strength by such men. Then they parted with much affection.

The king went northwards through Halogaland as he had purposed, and returned south as summer wore on. He went to yet other banquets there that were prepared for him.

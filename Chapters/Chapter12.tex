\chapter{Hildirida's sons talk with Harold}
Hildirida's sons went to the king and bade him to a three nights' banquet. The king accepted their bidding, and fixed when he would come. So at the appointed time he and his train came thither. The company was not numerous, but the feast went off very well, and the king was quite cheerful. Harek entered into talk with the king, and their talk turned on this, that he asked about the king's journeys in those parts during the summer.

The king answered his questions, and said that all had received him well, each after his means.

`Great will have been the difference,' said Harek, `and at Torgar the company at the banquet will have been the most numerous.'

The king said that it was so.

Harek said: `That was to be looked for, because on that banquet most was spent; and thou, O king, hadst great luck in matters so turning out that thy life was not endangered. The end was as was likely; thou wert very wise and very fortunate; for thou at once suspectedst all was not for good on seeing the numerous company there gathered; but (as I am told) thou madest all thy men remain armed constantly and keep watch and ward night and day.'

The king looked at him and said: `Why speakest thou thus, Harek? What canst thou tell of this?'

Harek answered: `May I speak with permission what I please?'

`Speak,' said the king.

`This I judge,' said Harek, `that thou wouldst not deem it to be well, if thou, O king, heardest every one's words, what men say when speaking their minds freely at home, how they think that it is a tyranny thou exercisest over all people. But the plain truth is, O king, that to rise against thee the people lack nothing but boldness and a leader. Nor is it wonderful in a man like Thorolf that he thinks himself above everyone; he wants not for strength and comeliness; he keeps a guard round him like a king; he has wealth in plenty, even though he had but what is truly his, but besides that he holds others' property equally at his disposal with his own. Thou, too, hast bestowed on him large grants, and he had now made all ready to repay them with ill. For this is the truth that I tell thee: when it was learnt that thou wert coming north to Halogaland with no more force than three hundred men, the counsel of people here was that an army should assemble and take thy life, O king, and the lives of all thy force. And Thorolf was head of these counsels, and it was offered him that he should be king over the Halogalanders and Naumdalesmen. Then he went in and out of each firth and round all the islands, and got together every man he could find and every weapon, and it was no secret that this army was to muster for battle against king Harold. But the truth is, O king, that though thou hadst somewhat less force than those who met thee, yet the farmer folk took flight when they saw thy fleet. Then this counsel was adopted, to meet thee with friendly show and bid thee to a banquet: but it was intended, when thou wert well drunk and lying asleep, to attack thee with fire and weapon. And here is a proof whether I am rightly informed; ye were led into a granary because Thorolf was loth to burn up his new and beautiful hall; and a further proof is that every room was full of weapons and armour. But when all their devices against thee miscarried, then they chose the best course they could; they hushed up their former purpose. And I doubt not that all may deny this counsel, because few, methinks, know themselves guiltless, were the truth to come out. Now this is my counsel, O king, that thou keep Thorolf near thee, and let him be in thy guard, and bear thy standard, and be in the forecastle of thy ship; for this duty no man is fitter. Or if thou wilt have him to be a baron, then give him a grant southwards in the Firths, where are all his family: thou mayest then keep an eye on him, that he make not himself too great for thee. But the business here in Halogaland put thou into the hands of men who are moderate and will serve thee faithfully, and have kinsfolk here, men whose relatives have had the same work here before. We two brothers are ready and willing for such service as thou wilt use us in; our father long had the king's business here, and it prospered in his hands. It is difficult, O king, to place men as managers here, because thou wilt seldom come hither thyself. The strength of the land is too little to need thy coming with an army, yet thou must not come hither again with few followers, for there are here many disloyal people.'

The king was very angry at these words, but he spoke quietly, as was always his wont when he heard tidings of great import. He asked whether Thorolf were at home at Torgar. Harek said this was not likely.

`Thorolf,' said he, `is too wise to be in the way of thy followers, O king, for he must guess that all will not be so close but thou wilt get to know these things. He went north to Alost as soon as he heard that thou wert on thy way south.'

The king spoke little about this matter before other men; but it was easy to see that he inclined to believe the words that had been spoken.

After this the king went his way, Hildirida's sons giving him honourable escort with gifts at parting, while he promised them his friendship. The brothers made themselves an errand into Naumdale, and so went round about as to cross the king's path now and again; he always received their words well.

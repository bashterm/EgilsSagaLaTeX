\chapter{Thorgils goes to the king}
There was a man named Thorgils Yeller, a house-carle of Thorolf's, honoured above all the rest of his household; he had followed Thorolf in his roving voyages as fore-castle man and standard-bearer. He had been in Hafr's Firth, in the fleet of king Harold, and was then steering the very ship that Thorolf had used in his roving. Thorgils was strong of body and right bold of heart; the king had bestowed on him friendly gifts after the battle, and promised him his friendship. Thorgils was manager at Torgar, and bore rule there when Thorolf was not at home.

Before Thorolf went away this time he had counted over all the king's tribute that he had brought from the fells, and he put it in Thorgils' hand, bidding him convey it to the king, if he himself came not home before the king returned south. So Thorgils made ready a large ship of burden belonging to Thorolf, and put the tribute on board, and taking about twenty men sailed southward after the king, and found him in Naumdale.

But when Thorgils met the king he gave him greeting from Thorolf, and said that he was come thither with the Finns' tribute sent by Thorolf. The king looked at him, but answered never a word, and all saw that he was angry. Thorgils then went away, thinking to find a better time to speak with the king; he sought Aulvir Hnuf, and told him what had passed, and asked him if he knew what was the matter.

`That do I not,' said he; `but this I have marked, that, since we were at Leka, the king is silent every time Thorolf is mentioned, and I suspect he has been slandered. This I know of Hildirida's sons, that they were long in conference with the king, and it is easy to see from their words that they are Thorolf's enemies. But I will soon be certain about this from the king himself.'

Thereupon Aulvir went to the king, and said: `Here is come Thorgils Yeller thy friend, with the tribute which is thine; and the tribute is much larger than it has been before, and far better wares. He is eager to be on his way; be so good, O king, as to go and see it; for never have been seen such good gray furs.'

The king answered not, but he went to where the ship lay. Thorgils at once set forth the furs and showed them to the king. And when the king saw that it was true, that the tribute was much larger and better, his brows somewhat cleared, and Thorgils got speech with him. He brought the king some bearskins which Thorolf sent him, and other valuables besides, which he had gotten upon the fells. So the king brightened up, and asked tidings of the journey of Thorolf and his company. Thorgils told it all in detail.

Then said the king: `Great pity is it Thorolf should be unfaithful to me and plot my death.'

Then answered many who stood by, and all with one mind, that it was a slander of wicked men if such words had been spoken, and Thorolf would be found guiltless. The king said he would prefer to believe this. Then was the king cheerful in all his talk with Thorgils, and they parted friends.

But when Thorgils met Thorolf he told him all that had happened.

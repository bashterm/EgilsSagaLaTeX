\chapter{Thorolf again in Finmark}
That winter Thorolf went again to Finmark, taking with him about a hundred men. As before, he held a fair with the Finns, and travelled far and wide over Finmark. But when he reached the far east, and his coming was heard of, then came to him some Kvens, saying that they were sent by Faravid, king of Kvenland, because the Kiriales were harrying his land; and his message was that Thorolf should go thither and bear him help; and further that Thorolf should have a share of the booty equal to the king's share, and each of his men as much as two Kvens. With the Kvens the law was that the king should have one-third as compared with his men when the booty was shared, and beyond that, as reserved for him, all bearskins and sables. Thorolf put this proposal before his men, giving them the choice to go or not; and the more part chose to venture it, as the prize was so great. This is was decided that they should go eastwards with the messengers.

Finmark is a wide tract; it is bounded westwards by the sea, wherefrom large firths run in; by sea also northwards and round to the east; but southwards lies Norway; and Finmark stretches along nearly all the inland region to the south, as also does Halogaland outside. But eastwards from Naumdale is Jamtaland, then Helsingjaland and Kvenland, then Finland, then Kirialaland; along all these lands to the north lies Finmark, and there are wide inhabited fell-districts, some in dales, some by lakes. The lakes of Finmark are wonderfully large, and by the lakes there are extensive forests. But high fells lie behind from end to end of the Mark, and this ridge is called Keels.

But when Thorolf came to Kvenland and met king Faravid, they made them ready for their march, being three hundred of the kings men and a fourth hundred Norsemen. And they went by the upper way over Finmark, and came where the Kiriales were on the fell, the same who had before harried the Kvens. These, when they were aware of the enemy, gathered themselves and advanced to meet them, expecting victory as heretofore. But, on the battle being joined, the Norsemen charged furiously forwards, bearing shields stronger than those of the Kvens; the slaughter turned to be in the Kiriales' ranks many fell, some fled. King Faravid and Thorolf took there immense wealth of spoil, and returned to Kvenland, whence afterwards Thorolf and his men came to Finmark, he and Faravid parting in friendship.

Thorolf came down from the fell to Vefsnir; then went first to his farm at Sandness, stayed there awhile, and in spring went with his men north to Torgar.

But when he came there, it was told him how Hildirida's sons had been that winter at Throndheim with king Harold, and that they would not spare to slander Thorolf with the king; and it was much questioned what grounds they had had for their slander. Thorolf answered thus: 1The king will not believe this, though such lies be laid before him; for there are no grounds for my turning traitor to him, when he has done me much good and no evil. And so far from wishing to do him harm (though I had the choice), I would much rather be a baron of his than be called king, when some other fellow-countrymen might rise and make me his thrall.'

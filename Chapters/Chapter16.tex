\chapter{Thorolf and the king}
In the summer Thorolf went south to king Harold at Throndheim, taking with him all the tribute and much wealth besides, and ninety men well arrayed. When he came to the king, he and his were placed in the guest-hall and entertained magnificently.

On the morrow Aulvir Hnuf went to his kinsman Thorolf; they talked together, Aulvir saying that Thorolf was much slandered, and the king gave ear to such tales. Thorolf asked Aulvir to plead his cause with the king, `for,' said he, `I shall be short-spoken before the king if he choose rather to believe the lies of wicked men than truth and honesty which he will find in me.'

The next day Aulvir came to see Thorolf, and told him he had spoken on his business with the king; `but,' said he, `I know no more than before what is in his mind.'

`Then must I myself go to him,' said Thorolf.

He did so; he went to the king where he sat at meat, and when he came in he greeted the king. The king accepted his greeting, and bade them serve him with drink. Thorolf said that he had there the tribute belonging to the king from Finmark; `and yet a further portion of booty have I brought as a present to thee, O king. And what I bring will, I know, owe all its worth to this, that it is given out of gratitude to thee.'

The king said that he could expect nought but good from Thorolf, `because,' said he, `I deserve nought else; yet men tell two tales of thee as to thy being careful to win my approval.'

`I am not herein justly charged,' said Thorolf, `if any say I have shown disloyalty to thee. This I think, and with truth: That they who speak such lying slanders of me will prove to be in nowise thy friends, but it is quite clear that they are my bitter enemies; `tis likely, however, that they will pay dearly for it if we come to deal together.'

Then Thorolf went away.

But on the morrow Thorolf counted out the tribute in the king's presence; and when it was all paid, he then brought out some bearskins and sables, which he begged the king to accept. Many of the bystanders said that this was well done and deserved friendship. The king said that Thorolf had himself taken his own reward. Thorolf said that he had loyally done all he could to please the king. `But if he likes it not,' said he, `I cannot help it: the king knows, when I was with him and in his train, how I bore myself; it is wonderful to me if the king thinks me other now than he proved me to be then.'

The king answered: `Thou didst bear thyself well, Thorolf, when thou wert with us; and this, I think, is best to do still, that thou join my guard, bear my banner, be captain over the guard; then will no man slander thee, if I can oversee night and day what thy conduct is.'

Thorolf looked on either hand where stood his house-carles; then said he: `Loth were I to deliver up these my followers: about thy titles and grants to me, O king, thou wilt have thine own way, but my following I will not deliver up while my means last, though I manage at my own sole cost. My request and wish, O king, is this, that thou come and visit me at my home, and the hear word of men whom thou trustest, what witness they bear to me in this matter; thereafter do as thou findest proof to warrant.'

The king answered and said that he would not again accept entertainment from Thorolf; so Thorolf went out, and made ready to return home.

But when he was gone, the king put into the hands of Hildirida's sons his business in Halogaland which Thorolf had before had, as also the Finmark journey. The king claimed ownership of the estate at Torgar, and of all the property that Brynjolf had had; and all this he gave into the keeping of Hildirida's sons. The king sent messengers with tokens to Thorolf to tell him of this arrangement, whereupon Thorolf took the ships belonging to him, put on board all the chattels he could carry, and with all his men, both freedmen and thralls, sailed northwards to his farm at Sandness, where he kept up no fewer and no less state than before.

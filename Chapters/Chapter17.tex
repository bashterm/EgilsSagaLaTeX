\chapter{Hildirida's sons in Finmark and at Harold's court}
Hildirida's sons took the business in Halogaland; and none gainsaid this because of the king's power, but Thorolf's kinsmen and friends were much displeased at the change. The two brothers went on the fell in the winter, taking with them thirty men. To the Finns there seemed much less honour in these stewards than when Thorolf came, and the money due was far worse paid.

That same winter Thorolf went up on the fell with a hundred men; he passed on at once eastwards to Kvenland and met king Faravid. They took counsel together, and resolved to go on the fell again as in the winter before; and with four hundred men they made a descent on Kirialaland, and attacked those districts for which they thought themselves a match in numbers, and harrying there took much booty, returning up to Finmark as the winter wore on. In the spring Thorolf went home to his farm, and then employed his men at the fishing in Vagar, and some in herring-fishing, and had the take of every kind brought to his farm.

Thorolf had a large ship, which was waiting to put to sea. It was elaborate in everything, beautifully painted down to the sea-line, the sails also carefully striped with blue and red, and all the tackling as elaborate as the ship. Thorolf had this ship made ready, and put aboard some of his house-carles as crew; he freighted it with dried fish and hides, and ermine and gray furs too in abundance, and other peltry such as he had gotten from the fell; it was a most valuable cargo. This ship he bade sail westwards for England to buy him clothes and other supplies that he needed; and they, first steering southwards along the coast, then stretching across the main, came to England. There they found a good market, laded the ship with wheat and honey and wine and clothes, and sailing back in autumn with a fair wind came to Hordaland.

That same autumn Hildirida's sons carried tribute to the king. But when they paid it the king himself was present and saw. He said:

`Is this tribute now paid all that ye took in Finmark?'

`It is,' they answered.

`Less by far,' said the king, `and much worse paid is the tribute now than when Thorolf gathered it; yet ye said that he managed the business ill.'

`It is well, O king,' said Harek, `that thou hast considered how large a tribute should usually come from Finmark, because thus thou knowest how much thou losest, if Thorolf waste all the tribute before thee. Last winter we were in Finmark with thirty men, as has been the wont of thy stewards heretofore. Soon after came Thorolf with a hundred men, and we learnt this, that he meant to take the lives of us two brothers and all our followers, his reason being that thou, O king, hadst handed over to us the business that he wished to have. It was then our best choice to shun meeting him, and to save ourselves: therefore we quickly left the settled districts, and went on the fell. But Thorolf went all round Finmark with his armed warriors; he had all the trade, the Finns paid him tribute, and he hindered thy stewards from entering Finmark. He means to be made king over the north there, both over Finmark and Halogaland: and the wonder is that thou wilt listen to him in anything whatever. Herein may true evidence be found of Thorolf's ill-gotten gains from Finmark; for the largest merchant ship in Halogaland was made ready for sea at Sandness in the spring, and all the cargo on board was said to be Thorolf's. It was laden mostly, I think, with gray furs, but there would be found there also bearskins and sables more than Thorolf brought to thee. And with that ship went Thorgils Yeller, and I believe he sailed westwards for England. But if thou wilt know the truth of this, set spies on the track of Thorgils when he comes eastwards; for I fancy that no trading-ship in our days has carried such store of wealth. And I am telling thee what is true, O king, when I say that to thee belongs every penny on board.'

All that Harek said his companions confirmed, and none there ventured to gainsay.

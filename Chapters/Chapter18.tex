\chapter{Thorolf's ship is taken}
There were two brothers named Sigtrygg Swiftfarer and Hallvard Hardfarer, kinsmen of king Harold on the mother's side; from their father, a wealthy man, they had inherited an estate in Hising. Four brothers there were in all; but Thord and Thorgeir, the two younger, were at home, and managed the estate. Sigtrygg and Hallvard carried all the king's messages, both within and without the land, and had gone on many dangerous journeys, both for putting men out of the way and confiscating the goods of those whose homes the king ordered to be attacked. They kept about them a large following; they were not generally in favour, but the king prized them highly. None could match them at travelling, either on foot or on snow-shoes; in voyaging also they were speedier than others, valiant men they were, and very wary.

These two men were with the king when those things happened that have just been told. In the autumn the king went to a banquet in Hordaland. And one day he summoned to him the brothers Hallvard and Sigtrygg, and when they came he bade them go with their following and spy after the ship which Thorgils had taken westward to England in the summer.

`Bring me,' said he, `the ship and all that is in it, except the men; let them go their way in peace, if they do not try to defend the ship.'

The brothers made them ready for this, and, taking each one his long-ship, went to seek Thorgils, and learnt that he was come from the west, and had sailed northwards along the coast. Northwards after him went they, and found him in Fir Sound. They knew the ship at once, and laid one of their ships on the seaward side of her, while some of them landed, and thence went out on to the ship by the gangways. Thorgils' crew, apprehending no danger, made no defence; they found out nothing till many armed men were aboard, and so they were all seized, and afterwards put on shore weaponless, with nothing but the clothes they wore. But Hallvard's men drew out the gangways, loosed the cables, and towed out the ship; then turned them about, and sailed southwards along the coast till they met the king, to whom they brought the ship and all that was in it. And when the cargo was unloaded, the king saw that it was great wealth, and what Harek had said was no lie.

But Thorgils and his comrades got conveyance, and went to Kveldulf and his son, and told of the misadventure of their voyage, yet were they well received. Kveldulf said all was tending to what he had foreboded, that Thorolf would not in the end have good luck in his friendship with king Harold.

`And I care little,' said he, `for Thorolf's money loss in this, if worse does not come after; but I misdoubt, as before, that Thorolf will not rightly rate his own means against the stronger power with which he has to deal.'

And he bade Thorgils say this to Thorolf:

`My counsel is that you go away out of the land, for maybe you will do better for yourself if you serve under the king of England, or of Denmark, or of Sweden.'

Then he gave Thorgils a rowing-cutter with tackling complete, a tent also, and provisions, and all things needful for their journey. So they departed, and stayed not their journey till they came to Thorolf and told him all that had happened.

Thorolf took his loss cheerfully, and said that he should not be short of money; `'tis good,' said he, `to be in partnership with a king.' He then bought meal and all that he needed for the maintenance of his people; his house-carles must for awhile, he said, be less bravely attired than he had purposed. Some lands he sold, some he mortgaged, but he kept up all expenses as before; he had no fewer men with him than last winter, nay, rather more. And as to feasts and friends entertained at his house, he had more means for all this than before. He stayed at home all that winter.

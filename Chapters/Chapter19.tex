\chapter{Thorolf retaliates}
When spring came, and the snow and ice were loosed, then Thorolf launched a large warship of his own, and he had it made ready, and equipped his house-carles, taking with him more than a hundred men; and a goodly company there were, and well weaponed. And when a fair wind blew, Thorolf steered southwards along the coast till he came to Byrda; then they held an outer course outside the islands, but at times through channels between hill-slopes. Thus they coasted on southwards, and had no tidings of men till they came eastwards to Vik. There they heard that king Harold was in Vik, meaning in the summer to go into Upland. The people of the country knew nothing of Thorolf's voyage. With a fair wind he held on south to Denmark, and thence into the Baltic, where he harried through that summer, but got no good booty. In the autumn he steered back from the east to Denmark, at the time when the fleet at Eyrar was breaking up. In the summer there had been, as was usual, many ships from Norway. Thorolf let all these vessels sail past, and did not show himself. One day at eventide he sailed into Mostrarsound , where in the haven was a large ship of burden that had come from Eyrar. The steersman was named Thorir Thruma; he was a steward of king Harold's, manager of his farm at Thruma, a large farm in which the king used to make a long stay when he was in Vik. Much provision was needed for this farm, and Thorir had gone to Eyrar for this, to buy a cargo, malt, wheat, and honey; and much wealth of the king's had he for that end. Thorolf made for this ship, and offered Thorir and his crew the choice to defend themselves, but, as they had no force to make defence against such numbers, they yielded. The ship with all its freight Thorolf took, but Thorir he put out on an island.

Then he sailed northwards along the coast with both the ships; but when they came to the mouth of the Elbe, they lay there and waited for night. And when it was dark, they rowed their long-ship up the river and stood in for the farm-buildings belonging to Hallvard and Sigtrygg. They came there before daybreak, and formed a ring of men round the place, then raised a war-whoop and wakened those within, who quickly leapt up to their weapons. Thorgeir at once fled from his bedchamber. Round the farmhouse were high wooden palings: at these Thorgeir leapt, grasping with his hand the stakes, and so swung himself out of the yard. Thorgils Yeller was standing near; he made a sweep with his sword at Thorgeir, and cut off his hand along with the fence-stake. Then Thorgeir escaped to the wood, but Thord, his brother, fell slain there, and more than twenty men. Thorolf's band plundered and burnt the house, then went back down the river to the sea.

With a fair wind they sailed north to Vik; there again they fell in with a large merchant-ship belonging to men of Vik, laden with malt and meal. For this ship they made; but those on board, deeming they had no means of defence, yielded, and were disarmed and put on shore, and Thorolf's men, taking the ship and its cargo, went on their way.

Thorolf had now three ships, with which he sailed westwards by Fold. Then they took the high road of the sea to Lidandisness, going with all despatch, but making raid and lifting cattle on ness and shore. Northwards from Lidandisness they held a course further out, but pillaged wherever they touched land. But when Thorolf came over against the Firths, then he turned his course inward, and went to see his father Kveldulf, and there they were made welcome. Thorolf told his father what had happened in his summer voyage; he stayed there but a short time, and Kveldulf and his son Grim accompanied him to the ship.

But before they parted Thorolf and his father talked together, and Kveldulf said: `I was not far wrong, Thorolf, in telling thee, when thou wentest to join king Harold's guard, that neither thou nor we thy kindred would in the long run get good-fortune therefrom. Now thou hast taken up the very counsel against which I warned thee; thou matchest thy force against king Harold's. But though thou art well endowed with valour and all prowess, thou hast not luck enough for this, to play on even terms with the king - a thing wherein no one here in the land has succeeded, though others have had great power and large force of men. And my foreboding is that this is our last meeting: it were in the course of nature from our ages that thou shouldst overlive me, but I think it will be otherwise.'

After this Thorolf embarked and went his way. And no tidings are told of his voyage till he arrived home at Sandness, and caused to be conveyed to his farm all the booty he had taken, and had his ship set up upon land. There was now no lack of provision to keep his people through the winter. Thorolf stayed on at home with no fewer men than in the winter before.

\chapter{Death of Thorolf Kveldulfsson}
King Harold was at Hlada when the brothers went away. Immediately after this the king made him ready with all haste, and embarked his force on four ships, and they rowed up the firth, and so by Beitis-sea inwards to the isthmus of Elda. There he left his ships behind, and crossed the isthmus northwards to Naumdale. The king there took ships belonging to the landowners, and embarked his force on them, having with him his guard; four hundred men they were. Six ships he had well equipped both with weapons and men. They encountered a fresh head-wind, and rowed night and day, making what progress they could. The night was then light enough for travel.

On the evening of a day after sunset they came to Sandness, and saw lying there opposite the farm a long-ship with tent spread, which they knew to be Thorolf's. He was even then purposing to sail away, and had bidden them brew the ale for their parting carousal. The king ordered his men to disembark and his standard to be raised. It was but a short way to the farm buildings.

Thorolf's watchmen sate within drinking, and were not gone to their posts; not a man was without; all sate within drinking. The king had a ring of men set round the hall: they then shouted a war-whoop, and a war-blast was blown on the king's trumpet. On hearing which Thorolf's men sprang to their weapons, for each man's weapons hung above his seat. The king caused some to make proclamation at the door, bidding women, children, old men, thralls, and bondmen to come out. Then came out Sigridr the mistress, and with her the women that were within, and the others to whom permission was given. Sigridr asked if the sons of Kari of Berdla were there. They both came forward and asked what she would of them.

`Lead me to the king,' said she.

They did so. But when she came to the king, she said: `Will anything, my lord, avail to reconcile thee with Thorolf?'

The king answered, `If Thorolf will yield him to my mercy, then shall he have life and limb, but his men shall undergo punishment according to the charges against them.'

Upon this Aulvir Hnuf went to the room, and had Thorolf called to speak with him, and told him what terms the king offered them.

Thorolf answered that he would not take of the king compulsory terms or reconciliation. `Bid thou the king allow us to go out, and then leave we things to go their own course.'

The king said: `Set fire to the room; I will not waste my men by doing battle with him outside; I know that Thorolf will work us great man-scathe if he come out, though he has fewer men than we.'

So fire was set to the room, and it soon caught, because the wood was dry and the walls tarred and the roof thatched with birch-bark. Thorolf bade his men break up the wainscoting and get gable-beams, and so burst through the planking; and when they got the beams, then as many men as could hold on to it took one beam, and they rammed at the corner with the other beam-end so hard that the clasps flew out, and the walls started asunder, and there was a wide outlet.

First went out Thorolf, then Thorgils Yeller, then the rest one after another. Fierce then was the fight; nor for awhile could it be seen which had the better of it, for the room guarded the rear of Thorolf's force. The king lost many men before the room began to burn; then the fire attacked Thorolf's side, and many of them fell. Now Thorolf bounded forwards and hewed on either hand; small need to bind the wounds of those who encountered him. He made for where the king's standard was, and at this moment fell Thorgils Yeller. But when Thorolf reached the shield-wall, he pierced with a stroke the standard-bearer, crying, `Now am I but three feet short of my aim.' Then bore at him both sword and spear; but the king himself dealt him his death-wound, and he fell forward at the king's feet. The king called out then, and bade them cease further slaughter; and they did so.

After this the king bade his men go down to the ships. To Aulvir Hnuf and his brother he said:

`Take ye Thorolf your kinsman and give him honourable burial; bury also the other men who have fallen, and see to the binding of the wounds of those who have hope of life; but let none plunder here, for all this is my property.'

This said, the king went down to his ships, and most of his force with him; and when they were come on board men began to bind their wounds. The king went round the ship and looked at men's wounds; and when he saw a man binding a surface-wound, he said: `Thorolf gave not that wound; his weapon bites far otherwise; few, methinks, bind the wounds which he gave; and great loss have we in such men.'

As soon as day dawned the king had his sail hoisted, and sailed south as fast as he could. As the day wore on, they came upon many rowing-vessels in all the sounds between the islands; the forces on board them had meant to join Thorolf, for spies of his had been southwards as far as Naumdale, and far and wide about the islands. These had got to know how Hallvard and his brother were come from the south with a large force meaning to attack Thorolf. Hallvard's company had constantly met a head-wind, and had waited about in various havens till news of them had gone the upper way overland, and Thorolf's spies had become aware of it, and this gathering of force was on this account.

The king sailed before a strong wind till he came to Naumdale; there he left the ships behind, and went by land to Throndheim, where he took his own ships that he had left there, and thence stood out to Hlada. These tidings were soon heard, and reached Hallvard and his men where they lay. They then returned to the king, and their voyage was much mocked at.

The brothers Aulvir Hnuf and Eyvind Lambi remained awhile at Sandness and saw to the burial of the slain. To Thorolf's body they gave all the customary honours paid at the burial of a man of wealth and renown, and set over him a memorial stone. They saw also to the healing of the wounded. They arranged also the house with Sigridr; all the stock remained, but most of the house-furniture and table-service and clothing was burnt. And when this was done, they went south and came to king Harold at Throndheim, and were with him for awhile.

They were sad, and spoke little with others. And it was so that one day the brothers went before the king, and Aulvir said:

`This permission we brothers claim of thee, O king, that we go home to our farms; for such things have happened here that we have no heart to share drink and seat with those who drew weapon on our kinsman Thorolf.'

The king looked at them, and answered curtly:

`I will not grant you this; ye shall be here with me.'

They went back to their place.

Next day, as the king sat in the audience hall, he had the brothers called to him, and said:

`Now shall ye know of that your business which ye began with me, craving to go home. Ye have been some while here with me, and have borne you well, and always done your duty. I have thought well of you in everything. Now will I, Eyvind, that thou go north to Halogaland. I will give thee in marriage Sigridr of Sandness, her that Thorolf had to wife; and I will bestow on thee all the wealth that belonged to Thorolf; thou shalt also have my friendship if thou canst keep it. But Aulvir shall remain with me; for his skill as skald I cannot spare him.'

The brothers thanked the king for the honour granted to them, and said that they would willingly accept it.

Then Eyvind made him ready for the journey, getting a good and suitable ship. The king gave him tokens for this matter. His voyage sped well, and he came north to Alost and Sandness. Sigridr welcomed him; and Eyvind then showed her the king's tokens and declared his errand, and asked her in marriage, saying that it was the king's message that he should obtain this match. But Sigridr saw that her only choice, as things had gone, was to let the king rule it. So the arrangement was made, and Eyvind married Sigridr, receiving with her the farm at Sandness and all the property that had been Thorolf's. Thus Eyvind was a wealthy man.

The children of Eyvind and Sigridr were Fid Squinter, father of Eyvind Skald-spoiler, and Geirlaug, whom Sighvat Red had to wife. Fid Squinter married Gunnhilda, daughter of earl Halfdan. Her mother was Ingibjorg, daughter of king Harold Fairhair. Eyvind Lambi kept the king's friendship so long as they both lived.

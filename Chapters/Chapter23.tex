\chapter{The slaying of Hildirida's sons}
There was a man named Kettle H\ae ing, son of Thorkel earl of Naumdale, and of Hrafnilda daughter of Kettle H\ae ing of Hrafnista. He was a man of wealth and renown; he had been a fast friend of Thorolf Kveldulf's son, and was his near kinsman. He had been out on that expedition when forces gathered in Halogaland with intent to join Thorolf, as has been written above. But when king Harold went south, and men knew of Thorolf's slaying, then they called a gathering.

H\ae ing took with him sixty men, and turned to Torgar. Hildirida's sons were there, and few men with them. He went up to the farm, and made an attack on them; and there fell Hildirida's sons, and most of those who were there; and H\ae ing and his company took all the wealth they could lay hands on. After that H\ae ing took two ships of burden, the largest he could get, and put on board all the wealth belonging to him that he could carry; his wife and children also he took, and all the men that had been with him in the late work. And when they were ready and the wind blew fair, they sailed out to sea. A man named Baug, H\ae ing's foster-brother, of good family and wealthy, steered the second ship.

A few winters before Ingjolf and Hjorleif had gone to settle in Iceland; their voyage was much talked about, and 'twas said there was good choice of land there. So H\ae ing sailed west over the sea to seek Iceland. And when they saw land, they were approaching it from the south. But because the wind was boisterous, and the surf ran high on the shore, and there was no haven, they sailed on westwards along the sandy coast. And when the wind began to abate, and the surf to calm down, there before them was a wide river-mouth. Up this river they steered their ships, and lay close to the eastern shore thereof. That river is now called Thjors-river; its stream was then much narrower and deeper that it is now. They unloaded their ships, then searched the land eastward of the river, bringing their cattle after them. H\ae ing remained for the first winter on the eastern bank of the outer Rang-river.

But in the spring he searched the land eastwards, and then took land between Thjors-river and Mark-fleet, from fell to firth, and made his home at Hofi by east Rang-river. Ingunn his wife bare a son in this spring after their first winter, and the boy was named Hrafn. And though the house there was pulled down, the place continued to be called Hrafn-toft.

H\ae ing gave Baug land in Fleet-lithe, down from Mark-river to the river outside Breidabolstead; and he dwelt at Lithe-end. To his shipmates H\ae ing gave land or sold it for a small price, and these first settlers are called land-takers. H\ae ing had sons Storolf, Herjolf, Helgi, Vestar; they all had land. Hrafn was H\ae ing's fifth son. He was the first law-man in Iceland; he dwelt at Hofi after his father, and was the most renowned of H\ae ing's sons.

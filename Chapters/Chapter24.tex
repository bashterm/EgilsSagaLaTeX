\chapter{Kveldulf's grief}
Kveldulf heard of his son Thorolf's death, and so deeply grieved was he at the tidings that he took to his bed from sorrow and age. Skallagrim came often to him, and talked with him; he bade him cheer up. `Anything,' (he said) `was more fitting than to become worthless and lie bedridden; better counsel is it that we seek to avenge Thorolf. Maybe we shall come across some of those who took part in his slaying; but if not that, yet there will be men whom we can reach, and thereby displease the king.'

Kveldulf sang a stave:

{\centering\emph{
	`Thorolf in northern isle \\
	(O cruel Norns!) is dead: \\
	Too soon the Thunder-god \\
	Hath ta'en my warrior son. \\
	Thor's heavy wrestler, age, \\
	Holds my weak limbs from fray: \\
	Though keen my spirit spurs, \\
	No speedy vengeance mine.' \\
}}
King Harold went that summer to Upland, and in the autumn westwards to Valres, and as far as Vors. Aulvir Hnuf was with the king, and often spoke with him about whether he would pay atonement for Thorolf, granting to Kveldulf and Skallagrim money compensation, or such honour as would content them. The king did not altogether refuse this, if father and son would come to him. Whereupon Aulvir started northwards for the Firths, nor stayed his journey till he came one evening to these twain. They received him gratefully, and he remained there for some time. Kveldulf questioned Aulvir closely about the doings at Sandness when Thorolf fell, what doughty deeds Thorolf had wrought before he fell, who smote him with weapon, where he received most wounds, what was the manner of his fall. Aulvir told him all that he asked; and that king Harold gave him the wound that was alone enough for his bane, and that Thorolf fell forward at the very feet of the king.

Then answered Kveldulf: `Good is that thou tellest; for 'tis an old saw that he will be avenged who falls forward, and that vengeance will reach him who stands before him when he falls; yet is it unlikely that such good-fortune will be ours.'

Aulvir told father and son that he hoped, if they would go to the king and crave atonement, that it would be a journey to their honour; and he bade them venture this, adding many words to that end.

Kveldulf said he was too old to travel: `I shall sit at home,' said he.

`Wilt thou go, Grim?' said Aulvir.

`I think I have no errand thither,' said Grim; `I shall seem to the king not fluent in speech; nor do I think I shall long pray for atonement.'

Aulvir said that he would not need to do so: `We will do all the speaking for thee as well as we can.'

And seeing that Aulvir pressed this matter strongly, Grim promised to go when he thought he could be ready. He and Aulvir set them a time when Grim should come to the king. Then Aulvir went away first, and returned to the king.

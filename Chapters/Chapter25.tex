\chapter{Skallagrim's journey to the king}
Skallagrim made him ready for this journey, choosing out of his household and neighbours the strongest and doughtiest that were to be found. One was Ani, a wealthy landowner, another Grani, a third Grimolf and his brother Grim, house-carles these of Skallagrim, and the two brothers Thorbjorn Krum and Thord Beigaldi. These were called Thororna's sons; she dwelt hard by Skallagrim, and was of magic skill. Beigaldi was a coal-biter. There was a man named Thorir Giant, and his brother Thorgeir Earthlong, Odd Lonedweller, and Griss Freedman. Twelve there were for the journey, all stalwart men, and several of them shape-strong.

They took a rowing-ship of Skallagrim's, went southwards along the coast, stood in to Ostra Firth, then travelled by land up to Vors to the lake there; and, their course lying so that they must cross it, they got a suitable rowing-ship and ferried them over, whence they had not very far to go to the farm where the king was being entertained.

They came there at the time when the king was gone to table. Some men they found to speak with outside in the yard, and asked what was going on. This being told them, Grim begged one to call Aulvir Hnuf to speak with him. The man went into the room and up to where Aulvir sat, and said: `There be men here outside newly come, twelve together, if men one may call them, for they are liker to giants in stature and semblance than to mortal men.'

Aulvir at once rose and went out, for he knew who they were who had come. He greeted well his kinsman Grim, and bade him go with him into the room.

Grim said to his comrades: `'Tis the custom here that men go weaponless before the king; six of us shall go in, the other six shall bide without and keep our weapons.'

Then they entered, and Aulvir went up to the king, Skallagrim standing at his back. Aulvir was spokesman: `Here now is come Grim Kveldulf's son; we shall feel thankful to thee, O king, if thou make his journey hither a good one, as we hope it will be. Many get great honour from thee to whom less is due, and who are not nearly so accomplished as is he in every kind of skill. Thou wilt also do this because it is a matter of moment to me, if that is of any worth in thy opinion.'

Aulvir spoke fully and fluently, for he was a man ready of words. And many other friends of Aulvir went before the king and pleaded this cause.

The king looked round, and saw that a man stood at Aulvir's back taller than the others by a head, and bald.

`Is that Skallagrim,' asked the king, `that tall man?'

Grim said he guessed rightly.

`I will then,' said the king, `if thou cravest atonement for Thorolf, that thou become my liege-man, and enter my guard here and serve me. Maybe I shall so like thy service that I shall grant thee atonement for thy brother, or other honour not less than I granted him; but thou must know how to keep it better than he did, if I make thee as great a man as was he.'

Skallagrim answered: `It is well known how far superior to me was Thorolf in every point, and he got no luck by serving thee, O king. Now will I not take that counsel; serve thee I will not, for I know I should get no luck by yielding thee such service as I should wish and as would be worthy. Methinks I should fail herein more than Thorolf.'

The king was silent, and his face became blood-red. Aulvir at once turned away, and bade Grim and his men go out. They did so. They went out, and took their weapons, and Aulvir bade them begone with all haste. He and many with him escorted them to the water-side. Before parting with Skallagrim, Aulvir said:

`Kinsman, thy journey to the king ended otherwise than I would have chosen. I urged much thy coming hither; now, I entreat thee, go home with all speed, and come not in the way of king Harold, unless there be better agreement between you than now seems likely, and keep thee well from the king and from his men.'

Then Grim and his company went over the water; but Aulvir with his men, going to the ships drawn up by the water-side, so hacked them about that none was fit to launch. For they saw men coming down from the king's house, a large body well armed and advancing furiously. These men king Harold had sent after them to slay Grim. The king had found words soon after Grim went out, and said:

`This I see in that tall baldhead: that he is brim full of wolfishness, and he will, if he can reach them, work scathe on men whom we should be loth to lose. Ye may be sure, ye against whom he may bear a grudge, that he will spare none, if he get a chance. Wherefore go after him and slay him.'

Upon this they went and came to the water, and saw no ship there fit to launch. So they went back and told the king of their journey, and that Grim and his comrades would now have got clear over the lake.

Skallagrim went his way with his comrades till he reached home; he then told Kveldulf of this journey. Kveldulf showed him well pleased that Skallagrim had not gone to the king on this errand to take service under him; he still said, as before, that from the king they would get only loss and no amends. Kveldulf and Skallagrim spoke often of their plans, and on this they were agreed, that they would not be able to remain in the land any more than other men who were at enmity with the king, but their counsel must be to go abroad. And it seemed to them desirable to seek Iceland, for good reports were given about choice of land there. Already friends and acquaintances of theirs had gone thither - to wit, Ingolf Arnarson, and his companions - and had taken to them land and homestead in Iceland. Men might take land there free of cost, and choose their homestead at will.

So they quite settled to break up their household and go abroad.

Thorir Hroaldson had in his childhood been fostered with Kveldulf, and he and Skallagrim were about of an age, and as foster-brothers were dear friends. Thorir had become a baron of the king's at the time when the events just told happened, but the friendship between him and Skallagrim continued.

Early in the spring Kveldulf and his company made ready their ships. They had plenty of good craft to choose from; they made ready two large ships of burden, and took in each thirty able-bodied men, besides women and children. All the movable goods that they could carry they took with them, but their lands none dared buy, for fear of the king's power. And when they were ready, they sailed away: first to the islands called Solundir, which are many and large, and so scored with bays that few men (it is said) know all their havens.

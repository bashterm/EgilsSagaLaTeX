\chapter{Of Guttorm}
There was a man named Guttorm, son of Sigurd Hart. He was mother's brother to king Harold; also he had been his foster-father, and ruler over his forces, for the king was a child when he first came to the throne. Guttorm had commanded the army in all battles which Harold had fought to bring the land under his sway. But when Harold became sole king of all Norway, and sat in peace, then he gave to his kinsman Guttorm Westfold and East-Agdir, and Hringariki, and all the land that had belonged to Halfdan Swarthy his father. Guttorm had two sons and two daughters. His sons were named Sigurd and Ragnar; his daughters Ragnhildr and Aslaug.

Guttorm fell sick, and when near his end sent to king Harold, bidding him see to his children and his province. Soon after this he died. On hearing of his death, the king summoned Hallvard Hardfarer and his brother, and told them to go on a message for him eastwards to Vik, he being then at Throndheim. They made great preparations for their journey, choosing them men and the best ship they could get; it was the very ship they had taken from Thorgils Yeller. But when they were ready, the king told them their errand: they were to go eastwards to Tunsberg, the market town where Guttorm had resided. `Ye shall,' said the king, `bring to me Guttorm's sons, but his daughters shall be fostered there till I bestow them in marriage. I will find men to take charge of the province and foster the maidens.'

So the brothers started with a fair wind, and came in the spring eastwards to Vik and to Tunsberg, and there declared their errand. They took the sons of Guttorm, and much movable property, and went their way back. The wind was then somewhat slack, and their voyage slower, but nothing happened till they sailed northwards over the Sogn-sea, having now a good wind and bright weather, and being in merry mood.

\chapter{Slaying of Hallvard and Sigtrygg}
All through the summer Kveldulf and Skallagrim kept a look-out shorewards on the highway of vessels. Skallagrim was very sharp-sighted. He saw Hallvard's company sailing by, and he knew the ship, for he had seen it before when Thorgils went with it. Skallagrim watched their course, and where they lay to in haven at eventide. Then he went back to his own people, and told Kveldulf what he had seen, and withal how he had recognised the ship, being that which once was Thorolf's, and was taken by Hallvard from Thorgils, and doubtless there were some men on board who would be worth catching.

So they made them ready with both their boats, and twenty men in each. Kveldulf steered one, Skallagrim the other. Then they rowed and made for the ship. But when they came where it lay, they put in to land.

Hallvard's men had set up the tent over their ship, and laid them down to sleep. But when Kveldulf's force came upon them, then the watchmen who sat at the gangway-end leapt up, and called out to the ship; they bade the men rise, for an enemy was upon them. Hallvard's party leapt to their weapons. But when Kveldulf with his men came to the gangway-end, he went out by the stern gangway, while Skallagrim went forward to the other gangway.

Kveldulf had in his hand a battle-axe; but when he got on board, he bade his men go along the outer way by the gunwale and cut the tent from its forks, while he himself rushed aft to the stern-castle. And it is said that he then had a fit of shape-strength, as had also several of his comrades. They slew all that came in their way, the same did Skallagrim where he boarded the ship; nor did father and son stay hands till the ship was cleared. When Kveldulf came aft to the stern-castle, he brandished high his battle-axe, and smote Hallvard right through helm and head, so that the axe sank in even to the shaft; then he snatched it back towards him so forcibly that he whirled Hallvard aloft, and slung him overboard. Skallagrim cleared the forecastle, slaying Sigtrygg. Many men plunged into the sea; but Skallagrim's men took one of the boats, and rowed after and slew all that were swimming.

There were lost with Hallvard fifty men in all. The ship and the wealth that was in it Skallagrim's men took. Two or three men whom they deemed of least note they seized, and gave them their lives, asking of them who had been in the ship, and what had been the purport of the voyage. After learning all the truth about this, they looked over the slain who lay on ship-board. It was found that more had leapt overboard, and so perished, than had fallen on the ship. The sons of Guttorm had leapt overboard and perished. Of these, one was twelve years old, the other ten, and both were lads of promise.

Then Skallagrim set free the men whose lives he had spared, and bade them go to king Harold and tell him the whole tale of what had been done there, and who had been the doers of it. `Ye shall also,' said he, 'bear to the king this ditty:

{\centering\emph{
	`For a noble warrior slain
	Vengeance now on king is ta'en:
	Wolf and eagle tread as prey
	Princes born to sovereign sway.
	Hallvard's body cloven through
	Headlong in the billows flew;
	Wounds of wight once swift to fare
	Swooping vulture's beak doth tear.'
}}


After this Skallagrim and his men took out to their ships and captured ship and her cargo. And then they made an exchange, loading the ship they had taken, but emptying one of their own which was smaller; and in this they put stones, and bored holes and sank it. Then, as soon as ever the wind was fair, they sailed out to sea.

It is said of shape-strong men, or men with a fit of Berserk fury on them, that while the fit lasted they were so strong that nought could withstand them; but when it passed off, then they were weaker than their wont. Even so it was with Kveldulf. When the shape-strong fit went from him, then he felt exhaustion from the onset he had made, and became so utterly weak that he lay in bed.

And now a fair wind took them out to sea. Kveldulf commanded the ship which they had taken from Hallvard. With the fair wind the ships kept well together, and for long time were in sight of each other.

But when they were now far advanced over the main, Kveldulf's sickness grew worse. And when it came to this, that death was near, then he called to him his shipmates, and told them that he thought it likely they and he would soon take different ways. `I have never,' he said, `been an ailing man; but if it so be (as now seems likely) that I die, then make me a coffin, and put me overboard: and it will go far otherwise than I think if I do not come to Iceland and take land there. Ye shall bear my greeting to my son Grim, when ye meet, and tell him withal that if he come to Iceland, and things so turn out that unlikely as it may seem I be there first, then he shall choose him a homestead as near as may be to where I have come ashore.'

Shortly after this Kveldulf died.

His shipmates did as he had bidden them do; they laid him in a coffin, and shot it overboard. There was a man named Grim, son of Thorir Kettlesson Keel-fare, of noble kin and wealthy. He was in Kveldulf's ship; he had been an old friend of both father and son, and a companion both of them and of Thorolf, for which reason he had incurred the king's anger. He now took command of the ship after Kveldulf was dead.

But when they were come to Iceland, approaching the land from the south, they sailed westwards along the coast, because they had heard that Ingolf had settled there. But coming over against Reykja-ness, and seeing the firth open before them, they steered both ships into the firth.

And now the wind came on to blow hard, with much rain and mist. Thus the ships were parted.

Grim the Halogalander and his crew sailed in up the Borgar Firth past all the skerries; then they cast anchor till the wind fell and the weather cleared. They waited for the flood-tide, and then took their ship up into a river-mouth; it is called Gufu-river. They drew the ship up this river as far as it could go; then unshipped the cargo, and remained there for the first winter. They explored the land along the sea both inwards and outwards, and they had not gone far before they found Kveldulf's coffin cast up in a creek. They carried the coffin to the ness hard by, set it down there, and raised thereover a pile of stones.

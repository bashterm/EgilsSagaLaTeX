\chapter{Of Skallagrim's land-taking}
Skallagrim came to land where a large ness ran out into the sea, and above the ness was a narrow isthmus; and there they put out their lading. That ness they called Ship-ness. Then Skallagrim spied out the land: there was much moorland and wide woods, and a broad space between fells and firths, seal-hunting in plenty, and good fishing. But as they spied out the land southwards along the sea, they found before them a large firth; and, turning inwards along this firth, they stayed not their going till they found their companions, Grim the Halogalander and the rest. A joyful meeting was there. They told Skallagrim of his father's death, and how Kveldulf had come to land there, and they had buried him. Then they led Skallagrim to the place, and it seemed to him that thereabouts would be a good spot to build a homestead. He then went away, and back to his shipmates; and for that winter each party remained where they had come to land. Then Skallagrim took land between fells and firths, all the moors out to Seal-loch, and the upper land to Borgarhraun, and southwards to Hafnar-fell, and all that land from the watershed to the sea. Next spring he moved his ship southwards to the firth, and into the creek close to where Kveldulf came to land; and there he set his homestead, and called it Borg, and the firth Borgar-firth, and so too the country-side further up they named after the firth.

To Grim the Halogalander he gave dwelling-place south of Borgar-firth, on the shore named Hvann-eyrr. A little beyond this a bay of no great size cuts into the land. There they found many ducks, wherefore they called it Duck-kyle, and the river that fell into the sea there Duck-kyle-river. From this river to the river called Grims-river, the land stretching upwards between them Grim had. That same spring, as Skallagrim had his cattle driven inwards along the sea, they came to a small ness where they caught some swans, so they called it Swan-ness. Skallagrim gave land to his shipmates. The land between Long-river and Hafs-brook he gave to Ani, who dwelt at Anabrekka. His son was Aunund Sjoni. About this was the controversy of Thorstein and Tongue Odd.

Grani dwelt at Granastead on Digraness. To Thorbjorn Krum he gave the land by Gufu-river upward, and to Thord of Beigaldi. Krum dwelt at Krums-hills, but Thord at Beigaldi. To Thorir Giant and his brothers he gave land upwards from Einkunnir and the outer part by Long-river. Thorir Giant dwelt at Giantstead. His daughter was Thordis Staung, who afterwards dwelt at Stangerholt. Thorgeir dwelt at Earthlongstead.

Skallagrim spied out the land upwards all round the country-side. First he went inwards along the Borgar-firth to its head; then followed the west bank of the river, which he called White-river, because he and his companions had never before seen waters that fell out of glaciers, and the colour of the river seemed to them wonderful.

They went up along White-river till a river was before them coming down from the fells to the north; this they called North-river. And they followed it up till yet again before them was a river bringing down but little water. This river they crossed, and still went up along North-river; then they soon saw where the little river fell out of a cleft, and they called it Cleave-river. Then they crossed North-river, and went back to White-river, and followed that upwards. Soon again a river crossed their way, and fell into White-river; this they called Cross-river. They learnt that every river was full of fish. After this they returned to Borg.

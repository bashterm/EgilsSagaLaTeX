\chapter{Of Skallagrim's industry}
Skallagrim was most industrious. He had about him always many men, whom he set to seek diligently all such provisions as could be got there for man's sustenance, because at first they had but little live-stock compared with the needs of their numerous company. But what live-stock they had went every winter self-feeding in the woods.

Skallagrim was a good shipwright, and westwards of Myrar was no lack of driftwood. He had buildings set up on Swan-ness, and had another house there. This he made a starting-point for sea-fishing, seal-hunting, and egg-gathering; in all these kinds there was plenty of provisions to get, as well as driftwood to bring to him. Whales also often came in there, and whoso would might shoot them. All such creatures were then tame on the hunting-ground, as they were unused to man. His third house he had on the sea in Western Myrar. This was even a better place to look out for driftwood. There, too, he had land sown, and called it Acres. Over against it lay islands, among which whales were found; these they called Whale-islands.

Skallagrim also sent his men up on the salmon-rivers to fish. He set Odd Lonehouse by Cleave-river to see to the salmon-fishing there. Odd dwelt under Lonehouse. Lonehouse-ness has its name from him. Sigmund was the name of the man whom Skallagrim set by North-river; he dwelt at what was then called Sigmundstead, but now Hauga. Sigmundar-ness takes its name from him. He afterwards moved his homestead to Munodar-ness, that being thought more convenient for salmon-fishing.

But as Skallagrim's live-stock multiplied, the cattle used to go up to the fells in the summer. And he found that the cattle that went on the heath were by far better and fatter; also that sheep did well through the winters in the fell-dales without being driven down. So Skallagrim set up buildings close to the fell, and had a house there; and there he had his sheep kept. Of this farm Griss was the overlooker, and after him was called Grisartongue. Thus Skallagrim's wealth had many legs to stand on.

Some time after Skallagrim's coming out, a ship put into Borgar-firth from the main, commanded by a man named Oleif Halt. With him were his wife and children and other of his kin, and the aim of his voyage was to get him a home in Iceland. Oleif was a man wealthy, high-born, and fore-seeing. Skallagrim asked Oleif and all his company to his house for lodging. Oleif accepted this, and was with Skallagrim for his first winter in Iceland.

But in the following spring Skallagrim showed him to choice land south of White-river upwards from Grims-river to Flokadale-river. Oleif accepted this, and moved thither his household, and set there his homestead by Warm-brook as it is called. He was a man of renown; his sons were Ragi in Hot-spring-dale, and Thorarin, Ragi's brother, who took the law-speakership next after Hrafn H\ae ngsson. Thorarin dwelt at Warm-brook; he had to wife Thordis, daughter of Olaf Shy, sister of Thord Yeller.

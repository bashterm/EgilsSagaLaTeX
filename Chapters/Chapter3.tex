\chapter{The beginning of the rule of Harold Fairhair}
Harold, son of Halfdan Swarthy, was heir after his father. He had bound himself by this vow, not to let his hair be cut or combed till he were sole king over Norway, wherefore he was called Harold Shockhead. So first he warred with the kings nearest to him and conquered them, as is told at length elsewhere. Then he got possession of Upland; thence he went northwards to Throndheim, and had many battles there before he became absolute over all the Thronds. After that he purposed to go north to Naumdale to attack the brothers Herlaug and Hrollaug, kings of Naumdale. But when these brothers heard of his coming, Herlaug with twelve men entered the sepulchral mound which they had caused to be made (they were three winters at the making), and the mound then was closed after them. But king Hrollaug sank from royalty to earldom, giving up his kingdom and becoming a vassal of Harold. So Harold gained the Naumdalesmen and Halogaland, and he set rulers over his realm there. Then went he southwards with a fleet to M\ae ra and Raumsdale. But Solvi Bandy-legs, Hunthiof's son, escaped thence, and going to king Arnvid, in South M\ae ra, he asked help, with these words:

`Though this danger now touches us, before long the same will come to you; for Harold, as I ween, will hasten hither when he has enthralled and oppressed after his will all in North M\ae ra and Raumsdale. Then will the same need be upon you as was upon us, to guard your wealth and liberty, and to try everyone from whom you may hope for aid. And I now offer myself with my forces against this tyranny and wrong. But, if you make the other choice, you must do as the Naumdalesmen have done, and go of your own will into slavery, and become Harold's thralls. My father though it victory to die a king with honour rather than become in his old age another king's subject. Thou, as I judge, wilt think the same, and so will others who have any high spirit and claim to be men of valour.'

By such persuasion king Arnvid was determined to gather his forces and defend his land. He and Solvi made a league, and sent messengers to Audbjorn, king of the Firthfolk, that he should come and help them. Audbjorn, after counsel taken with friends, consented, and bade cut the war-arrow and send the war-summons throughout his realm, with word to his nobles that they should join him.

But when the king's messengers came to Kveldulf and told him their errand, and that the king would have Kveldulf come to him with all his house-carles, then answered he:

`It is my duty to the king to take the field with him if he have to defend his own land, and there be harrying against the Firthfolk; but this I deem clean beyond my duty, to go north to M\ae ra and defend their land. Briefly ye may say when ye meet your king that Kveldulf will sit at home during this rush to war, nor will he gather forces nor leave his home to fight with Harold Shockhead. For I think that he has a whole load of good-fortune where our king has not a handful.'

The messengers went back to the king, and told him how their errand had sped; but Kveldulf sat at home on his estates.

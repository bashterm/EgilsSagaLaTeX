\chapter{Of the coming out of Yngvar, and of Skallagrim's iron-forging}
King Harold Fair-hair took for his own all those lands that Kveldulf and Skallagrim had left behind in Norway, and all their other property that he could lay hands on. He also sought diligently after those men who had been in the counsels or confidence or in any way helpers of Skallagrim and his folk in the deeds which they wrought before Skallagrim went abroad out of the land. And so far stretched the enmity of the king against father and son, that he bore hatred against their kith and kin, or any whom he knew to have been their dear friends. Some suffered punishment from him, many fled away and sought refuge, some within the land, some out of the land altogether. Yngvar Skallagrim's wife's father was one of these men aforesaid. This rede did he take, that he turned all his wealth that he could into movables, then gat him a sea-going ship and a crew thereto, and made ready to go to Iceland, for he had heard that Skallagrim had taken up his abode there, and there would be no lack of choice land there with Skallagrim. So when they were ready and a fair wind blew, he sailed out to sea, and his voyage sped well. He came to Iceland on the south coast, and held on westwards past Reykja-ness, and sailed into Borgar-firth, and entering Long-river went up it even to the Falls. There they put out they ship's lading.

But when Skallagrim heard of Yngvar's coming, he at once went to meet him and bade him to his house with as many men as he would. Yngvar accepted this offer. The ship was drawn up, and Yngvar went to Borg with many men, and stayed that winter with Skallagrim. In the spring Skallagrim offered him choice land. He gave Yngvar the farm which he had on Swan-ness, and land inwards to Mud-brook and outwards to Strome-firth. Thereupon Yngvar went out to this farm and took possession, and he was a most able man and a wealthy. Skallagrim then built a house on Ship-ness, and this he kept for a long time thereafter.

Skallagrim was a good iron-smith, and in winter wrought much in red iron ore. He had a smithy set up some way out from Borg, close by the sea, at a place now called Raufar-ness. The woods he thought were not too far from thence. But since he could find no stone there so hard or smooth as he thought good for hammering iron on (for there are no beach pebbles, the seashore being all fine sand), one evening, when other were gone to sleep, Skallagrim went to the sea, and pushed out an eight-oared boat he had, and rowed out to the Midfirth islands. There he dropped an anchor from the bows of the boat, then stepped overboard, and dived down to the bottom, and brought up a large stone, and lifted it into the boat. Then he himself climbed into the boat and rowed to land, and carried the stone to the smithy and laid it down before the smithy door, and thenceforth he hammered iron on it. That stone lies there yet, and much slag beside it; and the marks of the hammering may be seen on its upper face, and it is a surf-worn boulder, unlike the other stones that are there. Four men nowadays could not lift a larger mass. Skallagrim worked hard at smithying, but his house-carles grumbled thereat, and thought it over early rising. Then Skallagrim composed this stave:

\begin{verse}
`Who wins wealth by iron\\
Right early must rise:\\
Of the sea's breezy brother\\
Wind-holders need blast.\\
On furnace-gold glowing\\
My stout hammer rings,\\
While heat-feeding bellows\\
A whistling storm stir.'\\
\end{verse}

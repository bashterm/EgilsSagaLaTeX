\chapter{Of Skallagrim's children}
Skallagrim and Bera had a great many children, but at first they all died. Then they had a son, who was sprinkled with water and named Thorolf. As a child he soon grew to be tall and was fair of countenance. It was the talk of all that he would be just such another as Thorolf Kveldulf's son, after whom he was named. Thorolf was far beyond children of his own age in strength. And as he grew to manhood he became doughty in most accomplishments then in vogue among those who were well trained. Thorolf was of a right cheery mood. Early did he come to such full strength as to be deemed fit for warlike service with other men. He was soon a favourite with all, and his father and mother loved him well. Skallagrim and his wife had two daughters; one was named S\ae unn, the other Thorunn. They also were of great promise as they grew up. Then Skallagrim and his wife had yet another son. He was sprinkled with water and named, and his name was Egil. But as he grew up it was soon seen that he would be ill-favoured, like his father, with black hair. When but three years old he was as tall and strong as other boys of six or seven. He was soon talkative and word-wise. Somewhat ill to manage was he when at play with other lads.

That spring, Yngvar went to Borg, his errand being to bid Skallagrim to a feast at his house, he also named for the party his daughter Bera and Thorolf her son, and any others that Skallagrim liked to bring. Skallagrim promised to come. Yngvar then went home, prepared for the banquet, and had ale brewed. But when the set time came that Skallagrim and Bera should go to the feast, Thorolf made him ready to go with them, as also some house-carles, so that they were fifteen in all. Egil told his father that he wished to go.

`I am,' said he, `as much akin to Yngvar as is Thorolf.'
`You shall not go,' said Skallagrim, `for you know not how to behave yourself in company where there is much drinking, you who are not good to deal with though you be sober.'

Then Skallagrim mounted his horse and rode away, but Egil was ill content with his lot.

He went out of the yard, and found a draught horse of Skallagrim's, got on its back and rode after Skallagrim's party. No easy way had he over the moor, for he did not know the road; but he kept his eyes on the riders before him when copse or wood were not in the way. And this is to tell of his journey, that late in the evening he came to Swan-ness, when men sat there a-drinking. He went into the room, but when Yngvar saw Egil he received him joyfully, and asked why he had come so late. Egil told of his words with Skallagrim. Yngvar made Egil sit by him, they two sat opposite Skallagrim and Thorolf. For merriment over their ale they fell to reciting staves. Then Egil recited a stave:

\begin{verse}
`Hasting I came to the hearth fire \\
Of Yngvar, right fain so to find him, \\
Him who on heroes bestoweth \\
Gold that the heather-worm guardeth. \\
Thou, of the snake's shining treasure \\
Always a generous giver, \\
Wilt not than me of three winters \\
Doughtier song-smith discover.' \\
\end{verse}


Yngvar praised this stave, and thanked Egil much therefor, but on the morrow he brought to Egil as reward for the poem three sea-snail shells and a duck's egg. And next day at the drinking Egil recited another stave about his poem's reward:

\begin{verse}
`The wielder of keen-biting wound-fowl \\
Gave unto Egil the talker \\
Three silent dogs of the surf-swell, \\
Meet for the praise in his poem. \\
He, the skilled guide of the sea-horse, \\
Knowing to please with a present, \\
Gave as fourth gift to young Egil \\
Round egg, the brook-bird's bed-bolster.' \\
\end{verse}

Egil's poetry won him thanks from many men. No more tidings were there of that journey. Egil went home with Skallagrim.

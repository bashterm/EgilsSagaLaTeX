\chapter{Of lord Brynjolf and Bjorn, his son}
There was in Sogn a lord named Bjorn, a rich man; he dwelt at Aurland. His son was Brynjolf, who was sole heir to all his father's wealth. Brynjolf's sons were Bjorn and Thord. They were young when what has been just told happened. Bjorn was a great traveller, sometimes on free-booting, sometimes on trading voyages. He was a right doughty man. It so chanced that one summer Bjorn was present at a banquet attended by many. He saw there a fair maiden who pleased him well. He asked of what family she was, and was told that she was sister of lord Thorir Hroaldsson, and was named Thora, with the by-name Lacehand. Bjorn made his suit and asked Thora to wife. But Thorir refused his offer, and with this they parted. But that same autumn Bjorn took men and went with a cutter well equipt northwards to the Firths, and came to Thorir's when he was not at home. Bjorn took Thora away thence, and home with him to Aurland. They two were there for the winter, and Bjorn would fain hold a wedding with her. Brynjolf his father ill liked what Bjorn had done; he thought there was dishonour therein, whereas there had been ere this long friendship between Thorir and Brynjolf.

`So far,' said he, `Bjorn, from your holding a wedding with Thora here in my house without the leave of her brother, she shall be here as well respected as if she were my daughter and your sister.' And all had to be as Brynjolf ordered in his household, whether Bjorn liked it well or ill. Brynjolf sent men to Thorir to offer him atonement and redress for what Bjorn had done. Thorir bade Brynjolf send Thora home; no atonement could there be else. But Bjorn would in no wise let her go away, though Brynjolf begged it. And so the winter wore on.

But when spring came, then Brynjolf and Bjorn were talking one day of their matters. Brynjolf asked what Bjorn meant to do. Bjorn said 'twas likeliest that he should go away out of the land.

`Most to my mind is it,' said he, `that you should give me a long-ship and crew therewith, and I go a free-booting.'

`No hope is there of this,' said Brynjolf, `that I shall put in your hands a warship and strong force, for I know not but you will go about just what is against my wish; why even now already I have enough trouble from you. A merchant-ship I will give you, and wares withal: go you then southwards to Dublin. That voyage is now most highly spoken of. I will get you a good crew.'

Bjorn said he would take this as his father willed. So he had a good merchant-ship made ready, and got men for it. Bjorn now made him ready for this voyage, but was some time about it. But when he was quite ready and a fair wind blew, he embarked on a boat with twelve men and rowed in to Aurland, and they went up to the homestead and to his mother's bower. She was sitting therein with many women. Thora was there. Bjorn said Thora must go with him, and they led her away. But his mother bade the women not dare to let them know this within in the hall: Brynjolf, she said, would be in a sad way if he knew it, and this would bring about great mischief between father and son. But Thora's clothes and trinkets were all laid there ready to hand, and Bjorn and his men took all with them.

Then they went that night out to their ship, at once hoisted their sail, and sailed out by the Sogn-sea, and so to the main. They had an ill wind, before which they must needs run, and were long tossed about on the main, because they were bent on shunning Norway at all hazards. And so it was that one day they were sailing off the east coast of Shetland during a gale, and brake their ship in making land at Moss-ey. They got out the cargo, and went into the town that was there, carrying thither all their wares, and they drew up their ship and repaired damages.

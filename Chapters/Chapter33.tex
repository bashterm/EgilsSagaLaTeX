\chapter{Bjorn goes to Iceland}
A little before winter came a ship from the south out of the Orkneys, with the tidings that a long-ship had come in autumn to those islands. Therein were messengers of king Harold, with this errand to earl Sigurd, that the king would have Bjorn Brynjolfsson slain wherever he might be found, and the same message Harold sent to the Southern Isles and even to Dublin. Bjorn heard these tidings, and withal that he was outlawed in Norway. Forthwith on reaching Shetland Bjorn had held his wedding with Thora, and through the winter they stayed at Moss-ey-town.

But in spring, as soon as ever the sea began to calm, Bjorn drew forth his ship, and made him ready with all speed. And when he was ready and got a wind, he sailed out to the main. They had a strong breeze, and were but little time out ere they came to the south coast of Iceland. The wind was blowing on the land; then it bore them westwards along the coast, and so out to sea. But when they got a shift of wind back again, then they sailed for the land. There was not a single man on board who had been in Iceland before. They sailed into a wondrous large firth, the wind bearing them towards its western shore. Land-wards nothing was seen but breakers and harbourless shore. Then they stood slant-wise across the wind as they might (but still eastwards), till a firth lay over against them, into which they sailed, till all the skerries and the surf were passed. Then they put in by a ness. An island lay out opposite this, and a deep sound was between them: there they made fast the ship. A bay ran up west of the ness, and above this bay stood a good-sized rocky hill.

Bjorn and some men with him got into a boat, Bjorn telling his comrades to beware of saying about their voyage aught that might work them trouble. They rowed to the buildings, and found there men to speak to. First they asked where they had come to land. The men told them that this was named Borgar-firth; that the buildings they saw were called Borg; that the goodman was Skallagrim.

Bjorn at once remembered about him, and he went to meet Skallagrim, and they talked together. Skallagrim asked who they were. Bjorn named himself and his father, but Skallagrim knew Brynjolf well, so he offered to Bjorn such help as he needed. This Bjorn accepted thankfully. Then Skallagrim asked what others there were in the ship, persons of rank. Bjorn said there was Thora, Hroald's daughter, sister of lord Thorir. Skallagrim was right glad for that, and said that it was his bounden duty to give to the sister of Thorir his own foster-brother such help as she needed or he could supply; and he bade her and Bjorn both to his house with all his shipmates. Bjorn accepted this. So the cargo was moved from the ship up to the homestead at Borg. There they set up their booths; but the ship was drawn up into the brook hard by. And where Bjorn's party had their booths is still called Bjorn's home-field. Bjorn and his shipmates all took up their abode with Skallagrim, who never had about him fewer than sixty stout fellows.

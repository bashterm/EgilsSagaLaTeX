\chapter{The journey to Bjarmaland}

Eric Bloodaxe now took a share in the realm. He held oversight in Hordaland and the Firths; he took and kept about him a body-guard. And one spring Eric Bloodaxe made ready to go to Bjarmaland, and chose him much people for that voyage. Thorolf betook him to this voyage with Eric, and was in the forecastle of his ship, and bare his standard. Thorolf was then taller and stronger than other men, and herein like his father. In that expedition befell much tidings. Eric had a great battle by the river Dvina in Bjarmaland, wherein he won the victory, as is told in the lays about him. And in that voyage he took Gunnhilda, daughter of Auzur Toti, and brought her home with him. Gunnhilda was above all women beautiful and shrewd, and of magic cunning. There was great intimacy between Thorolf and Gunnhilda. Thorolf ever spend the winters with Eric, the summers in freebooting.

The next tidings were that Thora Bjorn's wife fell sick and died. But some while after Bjorn took to him another wife; she was named Alof, the daughter of Erling the wealthy of Ostr. They two had a daughter named Gunnhilda.

There was a man named Thorgeir Thornfoot; he dwelt in Fenhring of Hordaland, at a place called Askr. He had three sons - one named Hadd, another Bergonund, the third Atli the short. Bergonund was beyond other men tall and strong, and he was grasping and ungentle; Atli the short was of small stature, square-built, of sturdy strength. Thorgeir was a very rich man, a devoted heathen worshipper, of magic cunning. Hadd went out freebooting, and was seldom at home.
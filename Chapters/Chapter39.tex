\chapter{Kettle Blund comes out to Iceland}

This had happened while Thorolf was away, that one summer a merchant-ship from Norway came into Borgar-firth. Merchant-ships used then commonly to be drawn up into rivers, brook-mouths, or ditches. This ship belonged to a man named Kettle, and by-named Blund; he was a Norwegian of noble kin and wealthy. His son, named Geir, who was then of full age, was with him in the ship. Kettle meant to make his home in Iceland; he came late in the summer. Skallagrim knew all about him, and offered him lodging for himself and all his company. This Kettle took, and was with Skallagrim for the winter. That winter Geir, Kettle's son, asked to wife Thorunn, Skallagrim's daughter, and the match was made, and Geir took her.

Next spring Skallagrim showed Kettle to land above Oleif's land, by White-river, from Flokadale-river mouth to Reykjadale-river mouth, and all the tongue that lay between the rivers up to Redgill, and all Flokadale above the slopes. Kettle dwelt at Thrandarholt; Geir at Geirs-lithe; he had another farm in Reykjadale at Upper Reykir. He was called Geir the wealthy; his sons were Blund-Kettle and Thorgeir-blund. A third was Hrisa-blund, who first dwelt at Hrisa.

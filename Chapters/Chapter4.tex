\chapter{Battle of king Harold and Audbjorn}
King Audbjorn went with his forces northwards to M\ae ra; there he joined king Arnvid and Solvi Bandy-legs, and altogether they had a large host. King Harold also had come from the north with his forces, and the armies met inside Solskel. There was fought a great battle, with much slaughter in either host. Of the M\ae rian forces fell the kings Arnvid and Audbjorn, but Solvi escaped, and afterwards became a great sea-rover, and wrought much scathe on Harold's kingdom, and was nicknamed Bandy-legs. On Harold's side fell two earls, Asgaut and Asbjorn, and two sons of earl Hacon, Grjotgard and Herlaug, and many other great men. After this Harold subdued South M\ae ra. Vemund Audbjorn's brother still retained the Firthfolk, being made king. It was now autumn, and king Harold was advised not to go south in autumn-tide. So he set earl Rognvald over North and South M\ae ra and Raumsdale, and kept a numerous force about himself.

That same autumn the sons of Atli set on Aulvir Hnuf at his home, and would fain have slain him. They had such a force that Aulvir could not withstand them, but fled for his life. Going northwards to M\ae ra, he there found Harold, and submitted to him, and went north with the king to Throndheim, and he became most friendly with him, and remained with him for a long time thereafter, and was made a skald.

In the winter following earl Rognvald went the inner way by the Eid-sea southwards to the Firths. Having news by spies of the movements of king Vemund, he came by night to Naust-dale, where Vemund was at a banquet, and, surrounding the house, burnt within it the king and ninety men. After that Karl of Berdla came to earl Rognvald with a long-ship fully manned, and they two went north to M\ae ra. Rognvald took the ships that had belonged to Vemund and all the chattels he could get. Kari of Berdla then went north to king Harold at Throndheim, and became his man.

Next spring king Harold went southwards along the coast with a fleet, and subdued firths and fells, and arranged for men of his own to rule them. Earl Hroald he set over the Firthfolk. King Harold was very careful, when he had gotten new peoples under his power, about barons and rich landowners, and all those whom he suspected of being at all likely to raise rebellion. Every such man he treated in one of two ways: he either made him become his liege-man, or go abroad; or (as a third choice) suffer yet harder conditions, some even losing life or limb. Harold claimed as his own through every district all patrimonies, and all land tilled or untilled, likewise all seas and freshwater lakes. All landowners were to be his tenants, as also all that worked in the forest, salt-burners, hunters and fishers by land and sea, all these owed him duty. But many fled abroad from this tyranny, and much waste land was then colonized far and wide, both eastwards in Jamtaland and Helsingjaland, and also the West lands, the Southern isles, Dublin in Ireland, Caithness in Scotland, and Shetland. And in that time Iceland was found.

\chapter{Of Egil's and Skallagrim's games}

Skallagrim took much pleasure in trials of strength and games; he liked to talk about such. Ball-play was then a common game. Plenty of strong men there were at that time in the neighbourhood, but not one of strength to match with Skallagrim. He was now somewhat stricken in years. There was a man named Thord, son of Grani, at Granastead, who was of great promise; he was then young; very fond he was of Egil, Skallagrim's son. Egil often engaged in wrestling; he was headstrong and hot-tempered, but all had the sense to teach their sons to give way to Egil. A game of ball was held at White-river-dale in the early winter, to which was a great gathering of people from all the country-side. Thither went many of Skallagrim's household to the game. Chief among them was Thord, Grani's son. Egil asked Thord to let him go with him to the game; he was then in his seventh winter. Thord let him do so, and Egil mounted behind him. But when they came to the play-meeting, then the men made up sides for the play. Many small boys had come there too, and they made up a game for themselves. For this also sides were chosen.

Egil was matched to play against a boy named Grim, son of Hegg, of Hegg-stead. Grim was ten or eleven years old, and strong for his age. But when they played together Egil got the worst of it. And Grim made all he could of his advantage. Then Egil got angry and lifted up the bat and struck Grim, whereupon Grim seized him and threw him down with a heavy fall, and handled him rather roughly, and said he would thrash him if he did not behave. But when Egil got to his feet, he went out of the game, and the boys hooted at him.

Egil went to Thord and told him what had been done. Thord said:

`I will go with you, and we will be avenged on them.'

He gave into his hands a halberd that he had been carrying. Such weapons were then customary. They went where the boys' game was. Grim had now got the ball and was running away with it, and the other boys after him. Then Egil bounded upon Grim, and drove the axe into his head, so that it at once pierced his brain. After this Egil and Thord went away to their own people. The Myramen ran to their weapons, and so did either party. Oleif Halt, with his following, ran to help the Borgarmen, who were thus far the larger number, and they parted without doing more. But hence arose a quarrel between Oleif and Hegg. They fought at Laxfit, by Grims-river; there seven men fell, but Hegg was wounded to death, and his brother Kvig fell. But when Egil came home, Skallagrim said little about it; but Bera said Egil had in him the makings of a freebooter, and that 'twould be well, so soon as he were old enough, to give him a long-ship. Then Egil made a stave:

\begin{verse}
`Thus counselled my mother, \\
For me should they purchase \\
A galley and good oars \\
To go forth a-roving. \\
So may I high-standing, \\
A noble barque steering, \\
Hold course for the haven, \\
Hew down many foemen.' \\
\end{verse}

When Egil was twelve years old, he was grown so big that there were but few men howso large and strong that he could not overcome in games. In his twelfth winter he was often at games. Thord Grani's son was then twenty years old; he was very strong. As the winter wore on, if often chanced that the two, Egil and Thord, were matched against Skallagrim. And once in the winter it so befell that there was ball-play at Borg, southwards in Sandvik. Thord and Egil were set against Skallagrim in the game; and he became weary before them, so that they had the best of it. But in the evening after sunset it began to go worse with Egil and his partner. Skallagrim then became so strong and he caught up Thord and dashed him down so violently that he was all bruised and at once got his bane. Then he seized Egil. Now there was a handmaid of Skallagrim's named Thorgerdr Brak, who had nursed Egil when a child; she was a big woman, strong as a man, and of magic cunning. Said Brak:

`Dost thou turn they shape-strength, Skallagrim, against thy son?'

Whereat Skallagrim let Egil loose, but clutched at her. She broke away and took to her heels with Skallagrim after her. So went they to the utmost point of Digra-ness. Then she leapt out from the rock into the water. Skallagrim hurled after her a great stone, which struck her between the shoulders, and neither ever came up again. The water there is now called Brakar-sound. But afterwards, in the evening, when they came home to Borg, Egil was very angry. Skallagrim and everybody else were set at table, but Egil had not yet come to his place. He went into the fire-hall, and up to the man who there had the overseeing of work and the management of moneys for Skallagrim, and was most dear to him. Egil dealt him his deathblow, then went to his seat. Skallagrim spoke not a word about it then, and thenceforward the matter was kept quiet. But father and son exchanged no word good or bad, and so that winter passed.

The next summer after this Thorolf came out, as was told above. And when he had been in Iceland one winter, in the spring following he made ready his ship in Brakar-sound. But when he was quite ready, then one day Egil went to his father, and asked him to give him an outfit.

`I wish,' said he, `to go out with Thorolf.'

Skallagrim asked if he had spoken at all on that matter with Thorolf. Egil said he had not. Skallagrim bade him do that first. But when Egil started the question with Thorolf, he said:

`'Tis not likely that I shall take you abroad with me; if your father thinks he cannot manage you here in his house, I have no confidence for this, to take you with me to foreign lands; for it will not do to show there such temper as you do here.`

`Maybe,' said Egil, `neither of us will go.'

In the night came on a furious gale, a south-wester. But when it was dark, and now flood-tide, Egil came where the ship lay. He went out on to the ship, and outside the tenting; he cut asunder the cables that were on the seaward side; then, hurrying back to land by the bridge, he at once shot out the bridge, and cut the cables that were upon land. Then the ship was driven out into the firth. But when Thorolf's men were aware that the ship was adrift, they jumped into the boat; but the wind was far too strong for them to get anything done. The ship drifted over to Duck-kyle, and on the islands there; but Egil went home to Borg.

And when people got to know of the trick that Egil had played, the more part blamed it. Egil said he should before long do Thorolf more harm and mischief if he would not take him away. But then others mediated between them, and the end was that Thorolf took Egil, and he went out with him that summer.

When Thorolf came on shipboard, at once taking the axe which Skallagrim had given into his hands, he cast it overboard into the deep so that it nevermore came up. Thorolf went his way in the summer, and his voyage sped well, and they came out to Hordaland. He at once stood northwards to Sogn. There it had happened in the winter that Brynjolf had fallen sick and died, and his sons had shared the heritage. Thord had Aurland, the estate on which his father had dwelt. He had become a liege-man of the king, and was made a baron. Thord's daughter was named Rannveig, the mother of Thord and Helgi, this Thord being father if Ingiridr whom king Olaf had to wife. Helgi was father of Brynjolf, father of Serk, Sogn, and Svein.

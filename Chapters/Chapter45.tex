\chapter{Flight of Egil}

Egil went in the night and sought the places where boats were. But wheresoever he came to the strand, men were always there before him. He went thus through the whole night, and found never a boat. But when day dawned, he was standing on a certain ness. He saw then another island, and between him and it lay a very wide sound. This was then his counsel: he took helmet, sword, and spear, breaking off the spear-shaft and casting it out into the sea; but the weapons he wrapped round in his cloak and made thereof a bundle which he bound on his back. Then he plunged into the water, nor stayed his swimming till he came to the island. It was called Sheppey; it was an island of no great size covered with brushwood. There were cattle on it, both sheep and oxen, belonging to Atla-isle. But when he came to the island, he wrung his clothes dry.

By this time it was broad daylight, and the sun was risen. King Eric had Atla-island well searched as soon as it was light; this took some time, the island being large, and Egil was not found. Then the king made them row to other islands and seek him. It was evening when twelve men rowed to Sheppey. They were to look for Egil, and had also to bring from thence some cattle for slaughter. Egil saw the boat coming to the island; he then lay down and hid himself in the brushwood before the boat came to land. They left three men behind with the boat; but nine went up, and they separated into three search parties, with three in each. But when a rise in the ground was between them and the boat, then Egil stood up (having before got his weapons ready), and made straight across for the sea, and then along the shore. They who guarded the boat were not aware of it till Egil was upon them. He at once smote one with a death-blow; but another took to his heels, and he had to leap up something of a bank. Egil followed him with a blow cutting off his foot. The third man leapt out into the boat, and pushed off with the pole. Egil drew the boat to him with the rope, and leapt out into it. Few blows were exchanged ere Egil slew him, and pushed him overboard. Then he took oars and rowed the boat away. He went all that night and the day after, nor stayed till he came to lord Thorir's.

As for Aulvir and his comrades, the king let them go in peace, as guiltless in this matter.

But the men who were in Sheppey were there for many nights, and killed cattle for food, and made a fire and cooked them, and piled a large fuel-heap on the side of the island looking towards Atla-isle, and set fire thereto, and let folk know their plight. When that was seen, men rowed out to them, and brought to land those who yet lived.

The king was by this time gone away; he went to another banquet.

But of Aulvir there is this to be told, that he reached home before Egil, and Thorolf and Thorir had come home even before that. Aulvir told the tidings, the slaying of Bard and the rest that had there befallen, but of Egil's goings since he knew nothing. Thorolf was much grieved hereat, as also was Arinbjorn; they thought that Egil would return nevermore. But the next morning Egil came home. Which when Thorolf knew, he rose up and went out to meet him, and asked in what way he had escaped, and what tidings had befallen in his journey. Then Egil recited this stave:

\begin{verse}
`From Norway king's keeping, \\
From craft of Gunnhilda, \\
So I freed me (nor flaunt I The feat overbold), \\
That three, whom but I wot not, \\
The warrior king's liege-men, \\
Lie dead, to the high hall \\
Of Hela downsped.' \\
\end{verse}

Arinbjorn spoke well of this work, and said to his father that he would be bound to atone Egil with the king.

Thorir said, `It will be the common verdict that Bard got his desert in being slain; yet hath Egil wrought too much after the way of his kin, in looking little before him and braving a king's wrath, which most men find a heavy burden. However, I will atone you, Egil, with the king for this time.'

Thorir went to find the king, but Arinbjorn remained at home and declared that one lot should befall them all. But when Thorir came to the king, he offered terms for Egil, his own bail, while the king should doom the fine. King Eric was very wroth, and it was hard to come to speech with him; he said that what his father had said would prove true - that family would never be trustworthy. He bade Thorir arrange it thus: `Though I accept some atonement, Egil shall not be long harboured in my realm. But for the sake of thy intercession, Thorir, I will take a money fine for this man.' The king fixed such fine as he thought fit; Thorir paid it all and went home.

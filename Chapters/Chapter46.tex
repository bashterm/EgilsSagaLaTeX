\chapter{Of Thorolf's and Egil's harrying}

Thorolf and Egil stayed that winter with Thorir, and were made much of. But in spring they got ready a large war-ship and gathered men thereto, and in summer they went the eastern way and harried; there won they much wealth and had many battles. They held on even to Courland, and made a peace for half a month with the men of the land and traded with them. But when this was ended, then they took to harrying, and put in at divers places. One day they put in at the mouth of a large river, where was an extensive forest upon land. They resolved to go up the country, dividing their force into companies of twelve. They went through the wood, and it was not long before they came to peopled parts. There they plundered and slew men, but the people fled, till at last there was no resistance. But as the day wore on, Thorolf had the blast sounded to recall his men down to the shore. Then each turned back from where they were into the wood. But when Thorolf mustered his force, Egil and his company had not come down; and the darkness of night was closing in, so that they could not, as they thought, look for him.

Now Egil and his twelve had gone through a wood and then saw wide plains and tillage. Hard by them stood a house. For this they made, and when they came there they ran into the house, but could see no one there. They took all the loose chattels that they came upon. There were many rooms, so this took them a long time. But when they came out and away from the house, an armed force was there between them and the wood, and this attacked them. High palings ran from the house to the wood; to these Egil bade them keep close, that they might not be come at from all sides. They did so. Egil went first, then the rest, one behind the other, so near that none could come between.

The Courlanders attacked them vigorously, but mostly with spears and javelins, not coming to close quarters. Egil's party going forward along the fence did not find out till too late that another line of palings ran along on the other side, the space between narrowing till there was a bend and all progress barred. The Courlanders pursued after them into this pen, while some set on them from without, thrusting javelins and swords through the palings, while others cast clothes on their weapons. Egil's party were wounded, and after that taken, and all bound, and so brought home to the farmhouse.

The owner of that farm was a powerful and wealthy man; he had a son grown up. Now they debated what they should do with their prisoners. The goodman said that he thought this were best counsel, to kill them one on the heels of another. His son said that the darkness of night was now closing in, and no sport was thus gotten by their torture; he bade them be let bide till the morning. So they were thrust into a room and strongly bound. Egil was bound hand and foot to a post. Then the room was strongly locked, and the Courlanders went into the dining-hall, ate, drank, and were merry.

Egil strained and worked at the post till he loosed it up from the floor. Then the post fell, and Egil slipped himself off it. Next he loosed his hands with his teeth. But when his hands were loose, he loosed therewith the bonds from his feet. And then he freed his comrades; but when they were all loosed they searched round for the likeliest place to get out. The room was made with walls of large wooden beams, but at one end thereof was a smooth planking. At this they dashed and broke it through. They had now come into another room; this too had walls of wooden beams. Then they heard men's voices below under their feet. Searching about they found a trapdoor in the floor, which they opened. Thereunder was a deep vault; down in it they heard men's voices. Then asked Egil what men were these. He who answered named himself Aki. Would he like to come up, asked Egil. Aki answered, they would like it much.

Then Egil and his comrades lowered into the vault the rope with which they had been bound, and drew up thence three men. Aki said that these were his two sons, and they were Danes, who had been made prisoners of war last summer.

`I was,' he said, `well treated through the winter, and had the chief care of the goodman's property; but the lads were enslaved and had a hard lot. In spring we made up our minds to run away, but were retaken. Then we were cast into this vault.'

`You must know all about the plan of this house,' said Egil; `where have we the best hope to get out?'

Aki said that there was another plank partition: 'Break you up that, you will then come into a corn-store, whereout you may go as you will.'

Egil's men did so; they broke up the planking, came into the granary, and thence out. It was pitch dark.

Then said Egil's comrades that they should hasten to the wood. But Egil said to Aki, `If you know the house here, you can show us the way to some plunder.'

Aki said there was no lack of chattels. `Here is a large loft in which the goodman sleeps; therein is no stint of weapons.'

Egil bade them go to that loft. But when they came to the staircase head they saw that the loft was open. A light was inside, and servants, who were making the beds. Egil bade some stay outside and watch that none came out. Egil ran into the loft, seized weapons, of which there was no lack. They slew all the men that were in there, and they armed themselves fully. Aki went to a trapdoor in the floor and opened it, telling them that they should go down by this to the store-room below. They got a light and went thither. It was the goodman's treasury; there were many costly things, and much silver. There the men took them each a load and carried it out. Egil took under his arm a large mead-cask, and bare it so.

But when they came to the wood, then Egil stopped, and he said:

`This our going is all wrong, and not warlike. We have stolen the goodman's property without his knowing thereof. Never ought that shame to be ours. Go we back to the house, and let him know what hath befallen.'

All spoke against that, saying they would make for the ship.

Egil set down the mead-cask, then ran off, and sped him to the house. But when he came there, he saw that serving-lads were coming out of the kitchen with dishes and bearing them to the dining-hall. In the kitchen (he saw) was a large fire and kettles thereon. Thither he went. Great beams had been brought home and lighted, as was the custom there, by setting fire to the beam-end and so burning it lengthwise. Egil seized a beam, carried it to the dining-hall, and thrust the burning end under the eaves, and so into the birch bark of the roof, which soon caught fire. Some fagot-wood lay hard by; this Egil brought and piled before the hall-door. This quickly caught fire. But those who sate drinking within did not find it out till the flame burst in round the roof. Then they rushed to the door; but there was no easy way out, both by reason of the fagot-wood, and because Egil kept the door, and slew most who strove to pass out either in the doorway or outside.

The goodman asked who had the care of the fire.

Egil answered, `He has now the care of the fire whom you yester-even had thought least likely; nor will you wish to bake you hotter than I shall kindle; you shall have soft bath before soft bed, such as you meant to give to me and my comrades. Here now is that same Egil whom you bound hand and foot to the post in that room you shut so carefully. I will repay you your hospitality as you deserve.'

At this the goodman thought to steal out in the dark, but Egil was near, and dealt him his death-blow, as he did to many others. Brief moment was it ere the hall so burned that it fell in. Most of those who were within perished.

But Egil went back to the wood, where he found his comrades, and they all went together to the ship. Egil said he would have the mead-cask which he carried as his own special prize; it proved to be full of silver. Thorolf and his men were overjoyed when Egil came down. They put out from land as soon as day dawned; Aki and his two sons were with Egil's following. They sailed in the summer, now far spent, to Denmark, where they lay in wait for merchant-ships, and plundered when they got the chance.

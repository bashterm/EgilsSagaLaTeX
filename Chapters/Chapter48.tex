\chapter{Of the banquet at earl Arnfid's}

Thorolf stood northwards with his force past Holland, and they put into a harbour there, as the wind drove them back. They did not plunder there. A little way up the country dwelt an earl named Arnfid. But when he heard that freebooters had come to land there, he sent his men to meet them with this errand, to know whether they wished for peace or war. Upon the messengers' coming to Thorolf with their errand, he said that they would not harry there, that there was no need to harry there or come with warshield, the land being not wealthy. The messengers went back to the earl, and told him the issue of their errand: but when the earl knew that he need not gather men for this cause, then he rode down without any armed force to meet the freebooters. When they met, all went well at the conference. The earl bade Thorolf to a banquet with him, and as many of his men as he would. Thorolf promised to go.

On the appointed day the earl had riding-horses sent down to meet them. Thorolf and Egil went, and they had thirty men with them. When they came to the earl, he received them well; they were led into the dining-hall. At once beer was brought in and given them to drink. They sate there till evening.

But before the tables were removed the earl said that they should cast lots to drink together in pairs, man and woman, so far as numbers would allow, but the odd ones by themselves. They cast then their lots into the skirt of a cloak, and the earl drew them out. The earl had a very beautiful daughter then in the flower of youth; the lot decreed that Egil should sit by her for the evening. She was going about the floor of the hall amusing herself. Egil stood up and went to the place in which the earl's daughter had sat during the day. But when all took their several seats, then the earl's daughter went to her place. She said in verse:

\begin{verse}
`Why sittest in my seat, youth? \\
Thou seldom sure hast given \\ 
To wolf his warm flesh-banquet. \\
Alone I will mine own. \\
O'er carrion course thou heard'st not \\
Croak hoarse the joying raven, \\
Nor wentest where sword-edges \\
In warfare madly met.' \\
\end{verse}

Egil took her, and set her down by him. He sang:

\begin{verse}
`With bloody brand on-striding \\
Me bird of bane hath followed: \\
My hurtling spear hath sounded \\
In the swift Vikings' charge. \\
Raged wrathfully our battle, \\
Ran fire o'er foemen's rooftrees; \\
Sound sleepeth many a warrior \\
Slain in the city gate.' \\
\end{verse}

They two then drank together for the evening, and were right merry. The banquet was of the best, on that day and on the morrow. Then the rovers went to their ships, they and the earl parting in friendship with exchange of gifts.

Thorolf with his force then stood for the Brenn-islands. At that time these were a great lair of freebooters, because through the islands sailed many merchant-ships. Aki went home to his farms, and his sons with him. He was a very wealthy man, owning several farms in Jutland. He and Thorolf parted with affection, and pledged them to close friendship. But as autumn came on, Thorolf and his men sailed northward along the Norway coast till they reached the Firths, then went to lord Thorir.

He received them well, but Arinbjorn his son much better, who asked Egil to be there for the winter. Egil took this offer with thanks. But when Thorir knew of Arinbjorn's offer, he called it rather a hasty speech. `I know not,' said he, `how king Eric may like that; for after the slaying of Bard he said that he would not have Egil be here in the land.'

`You, father, can easily manage this with the king,' said Arinbjorn, `so that he will not blame Egil's stay. You will ask Thorolf, your niece's husband, to be here; I and Egil will have one winter home.'

Thorir saw from this talk that Arinbjorn would have his way in this. So father and son offered Thorolf winter-home there, which he accepted. They were there through the winter with twelve men.

Two brothers there were named Thorvald Proud and Thorfid Strong, near kinsmen of Bjorn Yeoman, and brought up with him. Tall men they were and strong, of much energy and forward daring. They followed Bjorn so long as he went out roving; but when he settled down in quiet, then these brothers went to Thorolf, and were with him in his harrying; they were forecastle men in his ship. And when Egil took command of a ship, then Thorfid was his forecastle man. These brothers followed Thorolf throughout, and he valued them most of his crew.

They were of his company this winter, and sate next to the two brothers. Thorolf sate in the high seat over against Thorir, and drank with him; Egil sate as cup-mate over against Arinbjorn. At all toasts the cup must cross the floor.

Lord Thorir went in the autumn to king Eric. The king received him exceedingly well. But when they began to talk together, Thorir begged the king not to take it amiss that he had Egil with him that winter. The king answered this well; he said that Thorir might get from him what he would, but it should not have been so had any other man harboured Egil. But when Gunnhilda heard what they were talking of, then said she: `This I think, Eric, that 'tis now going again as it has gone often before; thou lendest easy ear to talk, nor bearest long in mind the ill that is done thee. And now thou wilt bring forward the sons of Skallagrim to this, that they will yet again smite down some of thy near kin. But though thou mayest choose to think Bard's slaying of no account, I think not so.'

The king answered: `Thou, Gunnhilda, more than others provokest me to savageness; yet time was when thou wert on better terms with Thorolf than now. However I will not take back my word about those brothers.'

`Thorolf was well here,' said she, `before Egil made him bad; but now I reckon no odds between them.'

Thorir went home when he was ready, and told the brothers the words of the king and of the queen.

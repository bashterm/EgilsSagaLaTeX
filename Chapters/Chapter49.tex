\chapter{Slaying of Thorvald Proud}

Eyvind Skreyja and Alf were the names of two brothers of Gunnhilda, sons of Auzur Toti. They were tall and strong, and great traders. They were then made much of by king Eric and Gunnhilda. Not generally liked were they; at this time they were young, but fully grown to manhood. It so befell in the spring that a great sacrifice was fixed to be held in the summer at Gaular. Here was the most renowned chief temple. Thither flocked numbers from the firths and from the fells, and from Sogn, and almost all the great men. King Eric went thither. Then spoke Gunnhilda with her brothers: `I would fain that you two should so manage matters in this crowded gathering, that ye get to slay one of the two sons of Skallagrim, or, better still, both.'

They said it should be done.

Lord Thorir made ready to go thither. He called Arinbjorn to speak with him. `Now will I,' said he, `go to the sacrifice, but I will not that Egil go thither. I know the craft of Gunnhilda, the vehemence of Egil, the power of the king; no easy task were it to watch these all at once. But Egil will not let himself be hindered, unless you stay behind. Now Thorolf and the rest of his company shall go with me; Thorolf shall sacrifice and pray for happiness for his brother as well as himself.'

Whereupon Arinbjorn told Egil that he meant to stay at home; `and you shall be with me,' said he.

Egil agreed that it should be so.

But Thorir and the rest went to sacrifice, and a very great multitude was there, and there was much drinking. Thorolf went with Thorir wheresoever he went, and they never were apart day or night. Eyvind told Gunnhilda that he could get no chance at Thorolf. She bade him then slay some one of Thorolf's men rather than let everything fail.

It chanced one evening, when the king had gone to rest, as had also Thorir and Thorolf, but Thorfid and Thorvald still sate up, that the two brothers Eyvind and Alf came and sat down by them, and were very merry. First they drank as one drinking-party; but presently it came to this, that each should drink half a horn, Eyvind and Thorvald being paired together to drink, and Alf and Thorfid.

Now as the evening wore on there was unfair drinking; next followed bandying of words, then insulting language. Then Eyvind jumped up, drew a sword, and thrust at Thorvald, dealing him a wound that was his death. Whereupon up jumped on either side the king's men and Thorir's house-carles. But men were all weaponless in there, because it was sanctuary. Men went between and parted them who were most furious; nor did anything more happen that evening.

Eyvind had slain a man on holy ground; he was therefore made accursed, and had to go abroad at once. The king offered a fine for the man; but Thorolf and Thorfid said they never had taken man-fine, and would not take this. With that they parted. Thorir and his company went home. King Eric and Gunnhilda sent Eyvind south to Denmark to king Harold Gormsson, for he might not now abide on Norwegian soil. The king received him and his comrades well: Eyvind brought to Denmark a large war-ship. He then appointed Eyvind to be his coastguard there against freebooters, for Eyvind was a right good warrior.

In the spring following that winter Thorolf and Egil made them ready to go again a-freebooting. And when ready, they again stood for the eastern way. But when they came to Vik, they sailed then south along Jutland, and harried there; then went to Friesland, where they stayed for a great part of the summer; but then stood back for Denmark. But when they came to the borderland where Denmark and Friesland meet, and lay by the land there, so it was that one evening when they on shipboard were preparing for sleep, two men came to Egil's ship, and said they had an errand to him. They were brought before him. They said that Aki the wealthy had sent them thither with this message: `Eyvind Skreyja is lying out off Jutland-side, and thinks to waylay you as you come from the south. And he has gathered such large force as ye cannot withstand if ye encounter it all at once; but he himself goes with two light vessels, and he is even now here close by you.'

But when these tidings came before Egil, at once he and his took down their tenting. He bade them go silently; they did so. They came at dawn to where Eyvind and his men lay at anchor; they set upon them at once, hurling both stones and spears. Many of Eyvind's force fell there; but he himself leapt overboard and got to land by swimming, as did all those of his men who escaped. But Egil took his ships, cargo, and weapons.

They went back that day to their own company, and met Thorolf. He asked wither Egil had gone, and where he had gotten those ships with which they came. Egil said that Eyvind Skreyja had had the ships, but they had taken them from him. Then sang Egil:

\begin{verse}
`In struggle sternly hard
We strove off Jutland-side:
Well did the warrior fight,
Warder of Denmark's realm.
Till, with his wights o'erborne,
Eastwards from wave-horse high
To swim and seek the sand
Swift Eyvind Skreyja leapt.'
\end{verse}

Thorolf said: `Herein ye have so wrought, methinks, that it will not serve us as our autumn plan to go to Norway.'

Egil said it was quite as well, though they should seek some other place.

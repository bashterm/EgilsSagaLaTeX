\chapter{Of Athelstan king of the English}

Alfred the Great ruled England, being of his family the first supreme king over England. That was in the days of Harold Fairhair, king of Norway. After Alfred, Edward his son was king in England. He was father of Athelstan the Victorious, who was foster-father of Hacon the Good. It was at this time of our story that Athelstan took the kingdom after his father. There were several brothers sons of Edward.

But when Athelstan had taken the kingdom, then those chieftains who had before lost their power to his forefathers rose in rebellion; now they thought was the easiest time to claim back their own, when a young king ruled the realm. These were Britons, Scots, and Irish. King Athelstan therefore gathered him an army, and gave pay to all such as wished to enrich themselves, both foreigners and natives.

The brothers Thorolf and Egil were standing southwards along Saxony and Flanders, when they heard that the king of England wanted men, and that there was in his service hope of much gain. So they resolved to take their force thither. And they went on that autumn till they came to king Athelstan. He received them well; he saw plainly that such followers would be a great help. Full soon did the English king decide to ask them to join him, to take pay there, and become defenders of his land. They so agreed between them that they became king Athelstan's men.

England was thoroughly Christian in faith, and had long been so, when these things happened. King Athelstan was a good Christian; he was called Athelstan the Faithful. The king asked Thorolf and his brother to consent to take the first signing with the cross, for this was then a common custom both with merchants and those who took soldiers' pay in Christian armies, since those who were `prime-signed' (as 'twas termed) could hold all intercourse with Christians and heathens alike, while retaining the faith which was most to their mind. Thorolf and Egil did this at the king's request, and both let themselves be prime-signed. They had three hundred men with them who took the king's pay.

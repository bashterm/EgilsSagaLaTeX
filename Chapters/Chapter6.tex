\chapter{Thorolf resolves to serve the king}
Thorolf Kveldulf's son and Eyvind Lambi came home from sea-roving in the autumn. Thorolf went to his father, and father and son had some talk together. Thorolf asked what had been the errand of the men whom Harold sent thither. Kveldulf said the king had sent them with this message, that Kveldulf or else one of his sons should become his man.

`How answeredst thou?' said Thorolf.

`I spake what was in my mind, that I would never take service with king Harold; and ye two will both do the same, if I may counsel: this I think will be the end, that we shall reap ruin from that king.'

`That,' said Thorolf, `is quite contrary to what my mind tells me, for I think I shall get from him much advancement. And on this I am resolved, to seek the king, and become his man; and this I have learnt for true, that his guard is made up of none but valiant men. To join their company, if they will have me, seems to me most desirable; these men are in far better case than all others in the land. And 'tis told me of the king that he is most generous in money gifts to his men, and not slow to give them promotion and to grant rule to such as he deems meet for it. Whereas I hear this about all that turn their backs upon him and pay him not homage with friendship, that they all become men of nought, some flee abroad, some are made hirelings. It seems wonderful to me, father, in a man so wise and ambitious as thou art, that thou wouldst not thankfully take the dignity which the king offered thee. But if thou thinkest that thou hast prophetic foresight of this, that we shall get misfortune from this king, and that he will be our enemy, then why didst thou not go to battle against him with that king in whose service thou wert before? Now, methinks it is most unreasonable neither to be his friend nor his enemy.'

`It went,' said Kveldulf, `just as my mind foreboded, that they marched not to victory who went northwards to fight with Harold Shockhead in M\ae ra; and equally true will this be, that Harold will work much scathe on my kin. But thou, Thorolf, wilt take thine own counsel in thine own business; nor do I fear, though thou enter into the company of Harold's guards, that thou wilt not be thought capable and equal to the foremost in all proofs of manhood. Only beware of this, keep within bounds, nor rival thy betters; thou wilt not, I am sure, yield to others overmuch.'

But when Thorolf made him ready to go, Kveldulf accompanied him down to the ship and embraced him, with wishes for his happy journey and their next merry meeting.

\chapter{Of Bjorgolf, Brynjolf, Bard, and Hildirida}
There was a man in Halogaland named Bjorgolf; he dwelt in Torgar. He was a baron, powerful and wealthy; in strength, stature, and kindred half hill-giant. He had a son named Brynjolf, who was like his father. Bjorgolf was now old, and his wife was dead; and he had given over into his son's hands all business, and found him a wife, Helga, daughter of Kettle H\ae ing of Hrafnista. Their son was named Bard; he soon grew to be tall and handsome, and became a right doughty man.

One autumn there was a banquet where many men were gathered, Bjorgolf and his son being there the most honourable guests. In the evening they were paired off by lot to drink together, as the old custom was. Now, there was at the banquet a man named Hogni, owner of a farm in Leka, a man of great wealth, very handsome, shrewd, but of low family, who had made his own way. He had a most beautiful daughter, Hildirida by name; and it fell to her lot to sit by Bjorgolf. They talked much together that evening, and the fair maiden charmed the old man. Shortly afterwards the banquet broke up.

That same autumn old Bjorgolf journeyed from home in a cutter of his own, with thirty men aboard. He came to Leka, and twenty of them went up to the house, while ten guarded the ship. When they came to the farm, Hogni went out to meet him, and made him welcome, invited him and his comrades to lodge there, which offer Bjorgolf accepted, and they entered the room. But when they had doffed their travelling clothes and donned mantles, then Hogni gave orders to bring in a large bowl of beer; and Hildirida, the daughter of the house, bare ale to the guests.

Bjorgolf called to him Hogni the goodman, and said, 'My errand here is this: I will have your daughter to go home with me, and will even now make with her a hasty wedding.'

Hogni saw no choice but to let all be as Bjorgolf would; so Bjorgolf bought her with an ounce of gold, and they became man and wife, and Hildirida went home with Bjorgolf to Torgar. Brynjolf showed him ill-pleased at this business. Bjorgolf and Hildirida had two sons; one was named Harek, the other H\ae rek.

Soon after this Bjorgolf died; but no sooner was he buried than Brynjolf sent away Hildirida and her sons. She went to her father at Leka, and there her sons were brought up. They were good-looking, small of stature, naturally shrewd, like their mother's kin. They were commonly called Hildirida's sons. Brynjolf made little count of them, and did not let them inherit aught of their father's. Hildirida was Hogni's heiress, and she and her sons inherited from him and dwelt in Leka, and had plenty of wealth. Bard, Brynjolf's son, and Hildirida's sons were about of an age.

Bjorgolf and his son Brynjolf had long held the office of going to the Finns, and collecting the Finns' tribute.

Northwards, in Halogaland is a firth called Vefsnir, and in the firth lies an island called Alost, a large island and a good, and in this a farm called Sandness. There dwelt a man named Sigurd, the richest man thereabouts in the north; he was a baron, and wise of understanding. He had a daughter named Sigridr; she was thought the best match in Halogaland, being his only child and sole heiress to her father. Bard Brynjolf's son journeyed from home with a cutter and thirty men aboard northwards to Alost, and came to Sigurd at Sandness. There he declared his business, and asked Sigridr to wife. This offer was well received and favourable answered, and so it came about that Bard was betrothed to the maiden. The marriage was to take place the next summer. Bard was then to come north for the wedding.

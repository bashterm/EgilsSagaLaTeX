\chapter{Of Bard and Thorolf}
King Harold had that summer sent word to the men of power that were in Halogaland, summoning to him such as had not come to him before. Brynjolf resolved to go, and with him Bard his son; and in the autumn they went southwards to Throndheim, and there met the king. He received them most gladly. Brynjolf was made a baron of the king's; the king also gave him large grants beside what he had before. He gave him withal the right of journey to the Finns, with the king's business on the fells and the Finn traffic. Then Brynjolf went away home to his estate, but Bard remained, and was made one of the king's guard.

Of all his guard the king most prized his skalds; they occupied the second high seat. Of these Audun Ill-skald sat innermost, being the oldest; he had been skald to Halfdan Swarthy, king Harold's father. Next to him sat Thorbjorn Raven, then Aulvir Hnuf, and next to him was placed Bard; he was there by-named Bard the White or Bard the Strong. He was in honour with everyone there, but between him and Aulvir Hnuf was a close friendship.

That same autumn came to king Harold Thorolf Kveldulf's son and Eyvind Lambi, Kari of Berdla's son, and they were well received. They brought thither a swift twenty-benched long-ship well manned, which they had before used in sea-roving. They and their company were placed in the guest-hall; but when they had waited there till they thought it a fit time to go before the king, Kari of Berdla and Aulvir Hnuf went in with them. They greeted the king. Then said Aulvir Hnuf, `Here is come Kveldulf's son, whom I told thee in the summer Kveldulf would send. His promise to thee will now stand fast; for here thou canst see true tokens that he will be thy friend in all when he hath sent his son hither to take service with thee, a stalwart man as thou mayest see. Now, this is the boon craved by Kveldulf and by us all, that thou receive Thorolf with honour and make him a great man with thee.'

The king answered his words well, promising that so he would do, `If,' said he, `Thorolf proves himself as accomplished in deed as he is right brave in look.'

After this Thorolf was made of the king's household, and one of his guard.

But Kari of Berdla and his son Eyvind Lambi went back south in the ship which Thorolf had brought north, and so home to Kari's farm. Thorolf remained with the king, who appointed him a seat between Aulvir Hnuf and Bard; and these three struck up a close friendship. And all men said of Thorolf and Bard that they were a well-matched pair for comeliness, stature, strength, and all doughty deeds. And both were in high favour with the king.

But when winter was past and summer came, then Bard asked leave to go and see to the marriage promised to him the summer before. And when the king knew that Bard's errand was urgent, he allowed him to go home. Then Bard asked Thorolf to go north with him, saying (as was true) that he would meet there many of his kin, men of renown, whom he had not yet seen or known. Thorolf thought this desirable, so they got leave from the king for this; then they made them ready, took a good ship and crew, and went their way.

When they came to Torgar, they sent word to Sigurd that Bard would now see to that marriage on which they had agreed the summer before. Sigurd said that he would hold to all that they had arranged; so they fixed the wedding-day, and Bard with his party were to come north to Sandness. At the appointed time Brynjolf and Bard set out, and with them many great men of their kin and connexions. And it was as Bard had said, that Thorolf met there many of his kinsmen that he had not known before. They journeyed to Sandness, and there was held the most splendid feast. And when the feast was ended, Bard went home with his wife, and remained at home through the summer, and Thorolf with him.

In the autumn they came south to the king, and were with him another winter. During that winter Brynjolf died; and when Bard learnt that the inheritance there was open for him, he asked leave to go home. This the king granted, and before they parted Bard was made a baron, as his father had been, and held of the king all those same grants that Brynjolf had held. Bard went home to his estate, and at once became a great chief; but Hildirida's sons got no more of the heritage than before. Bard had a son by his wife; he was named Grim. Meanwhile Thorolf was with the king, and in great honour.

\chapter{Battle in Hafr's Firth}
King Harold proclaimed a general levy, and gathered a fleet, summoning his forces far and wide through the land. He went out from Throndheim, and bent his course southwards, for he had heard that a large host was gathered throughout Agdir, Rogaland, and Hordaland, assembled from far, both from the inland parts above, and from the east out of Vik, and many great men were there met who purposed to defend their land from the king. Harold held on his way from the north, with a large force, having his guards on board. In the forecastle of the king's ship were Thorolf Kveldulfsson, Bard the White, Kari of Berdla's sons, Aulvir Hnuf and Eyvind Lambi, and in the prow were twelve Berserks of the king.

The fleets met south in Rogaland in Hafr's Firth. There was fought the greatest battle that king Harold had had, with much slaughter in either host. The king set his own ship in the van, and there the battle was most stubborn, but the end was that king Harold won the victory. Thorir Longchin, king of Agdir, fell there, but Kjotvi the wealthy fled with all his men that could stand, save some that surrendered after the battle. When the roll of Harold's army was called, many were they that had fallen, and many were sore wounded. Thorolf was badly wounded, Bard even worse; nor was there a man unwounded in the king's ship before the mast, except those whom iron bit not to wit the Berserks.

Then the king had his men's wounds bound up, and thanked them for their valour, and gave them gifts, adding most praise where he thought it most deserved. He promised them also further honour, naming some to be steersmen, others forecastle men, others bow-sitters. This was the last battle king Harold had within the land; after this none withstood him; he was supreme over all Norway.

The king saw to the healing of his men, whose wounds gave them hope of life, as also to the burial of the dead with all customary honours. Thorolf and Bard lay wounded. Thorolf's wounds began to heal, but Bard's proved mortal. Then Bard had the king called to him, and spoke thus:

`If it so be that I die of these wounds, then I would ask this of thee, that I may myself name my heir.'

To this when the king assented, then said he:

`I will that Thorolf my friend and kinsman take all my heritage, both lands and chattels. To him, also, will I give my wife and the bringing up of my son, because I trust him for this above all men.'

This arrangement he made fast, as the law was, with the leave of the king. Then Bard died, and was buried, and his death was much mourned. Thorolf was healed of his wounds, and followed the king, and had won great glory.

In the autumn the king went north to Throndheim. Then Thorolf asked to go north to Halogaland, to see after those gifts which he had received in the summer from his kinsman Bard. The king gave leave for this, adding a message and tokens that Thorolf should take all that Bard had given him, showing that the gift was with the counsel of the king, and that he would have it so. Then the king made Thorolf a baron, and granted him all the rights which Bard had had before, giving him the journey to the Finns on the same terms. He also supplied to Thorolf a good long-ship, with tackling complete, and had everything made ready for his journey thence in the best possible way. So Thorolf set out, and he and the king parted with great affection.

But when Thorolf came north to Torgar, he was well received. He told them of Bard's death; also how Bard had left him both lands and chattels, and her that had been his wife; then he showed the king's order and tokens. When Sigridr heard these tidings, she felt her great loss in her husband, but with Thorolf she was already well acquainted, and knew him for a man of great mark; and this promise of her in marriage was good, and besides there was the king's command. So she and her friends saw it to be the best plan that she should be betrothed to Thorolf, unless that were against her father's mind. Thereupon Thorolf took all the management of the property, and also the king's business.

Soon after this Thorolf started with a long-ship and about sixty men, and coasted northwards, till one day at eventide he came to Sandness in Alost; there they moored the ship. And when they had raised their tent, and made arrangements, Thorolf went up to the farm buildings with twenty men. Sigurd received him well, and asked him to lodge there, for there had been great intimacy between them since the marriage connection between Sigurd and Bard. Then Thorolf and his men went into the hall, and were there entertained. Sigurd sat and talked with Thorolf, and asked tidings. Thorolf told of the battle fought that summer in the south, and of the fall of many men whom Sigurd knew well, and withal how Bard his son-in-law had died of wounds received in the battle. This they both felt to be a great loss. Then Thorolf told Sigurd what had been the covenant between him and Bard before he died, and he declared also the orders of the king, how he would have all this hold good, and this he showed by the tokens.

After this Thorolf entered on his wooing with Sigurd, and asked Sigridr, his daughter, to wife. Sigurd received the proposal well; he said there were many reasons for this; first, the king would have it so; next, Bard had asked it; and further he himself knew Thorolf well, and thought it a good match for his daughter. Thus Sigurd was easily won to grant this suit; whereupon the betrothal was made, and the wedding was fixed for the autumn at Torgar.

Then Thorolf went home to his estate, and his comrades with him. There he prepared a great feast, and bade many thereto. Of Thorolf's kin many were present, men of renown. Sigurd also came thither from the north with a long-ship and a chosen crew. Numerously attended was that feast, and it was at once seen that Thorolf was free-handed and munificent. He kept about him a large following, whereof the cost was great, and much provision was needed; but the year was good, and needful supplies were easily found.

During that winter Sigurd died at Sandness, and Thorolf was heir to all his property; this was great wealth.

Now the sons of Hildirida came to Thorolf, and put in the claim which they thought they had on the property that had belonged to their father Bjorgolf. Thorolf answered them thus:

`This I knew of Brynjolf, and still better of Bard, that they were men so generous that they would have let you have of Bjorgolf's heritage what share they knew to be your right. I was present when ye two put in this same claim on Bard, and I heard what he thought, that there was no ground for it, for he called you illegitimate.'

Harek said that they would bring witnesses that their mother was duly bought with payment.

`It is true that we did not at first treat of this matter with Brynjolf our brother it was a case of sharing between kinsmen, but of Bard we hoped to get our dues in every respect, though our dealings with him were not for long. Now however this heritage has come to men who are in nowise our kin, and we cannot be altogether silent about our wrong; but it may be that, as before, might will so prevail that we get not our right of thee in this, if thou refuse to hear the witness that we can bring to prove us honourably born.'

Thorolf then answered angrily:

`So far am I from thinking  you legitimate heirs that I am told your mother was taken by force, and carried home as a captive.'

After that they left talking altogether.
